\begin{enumerate}
  \item \lstinline!np.array([0, 1, 2, 3, 4, 5])!
  \item \lstinline!np.array((0, 1, 2, 3, 4, 5))!
  \item \lstinline!np.array(range(0,6))!
  \item \lstinline!np.arange(0,6)!
  \item \lstinline!np.fromfunction(f, [6], dtype=int)!
  \item \lstinline!np.fromiter(natural_numbers(), int, count=6)!
\end{enumerate}

Die letzten vier Unteraufgaben lassen sich beispielsweise mit einer Schleife
l�sen. Zu beachten ist hierbei lediglich, dass \lstinline!array.empty()!,
\lstinline!array.ones()! und \lstinline!array.zeros()! standardm��ig Arrays mit
Flie�kommazahlen erzeugen.
\begin{lstlisting}
a = np.empty(6, dtype=int)
b = np.ones(6, dtype=int)
c = np.zeros(6, dtype=int)
d = np.full(6, 0, dtype=int)
for i in range(0,6):
    a[i] = b[i] = c[i] = d[i] = i
print(a, b, c, d)
\end{lstlisting}
