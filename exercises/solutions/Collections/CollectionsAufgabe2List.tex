\begin{enumerate}
	\item Zun�chst legen wir eine List an und initialisieren diese mit den Werten von 1 bis 10, in beliebiger Reihenfolge.
	\lstinputlisting[language=Python, firstline=1,lastline=1]{exercises/src/Collections/CollectionsAufgabe2List.py}
	
	\item Die Summe wird mit Hilfe der Variable \lstinline$total$ und einer \lstinline$for$-Schleife ausgeben.
	\lstinputlisting[language=Python, firstline=3,lastline=7]{exercises/src/Collections/CollectionsAufgabe2List.py}
	
	\item Das Ausgeben des gr��ten Zahlwerts auf der Konsole. Hierf�r wird durch eine \lstinline$for$-Schleife jedes Element der List mit dem aktuellen \lstinline$maxValue$ verglichen und der gr��ere Wert gespeichert. 
	\lstinputlisting[language=Python, firstline=9,lastline=14]{exercises/src/Collections/CollectionsAufgabe2List.py} 
	
	\item Da die ersten beiden Zahlen gegeben sind, m�ssen acht weitere erzeugt werden. Hierf�r verwenden wir eine \lstinline$for$-Schleife und die \lstinline$range$-Methode. Diese generiert eine List mit den Zahlen des vorgegebenen Intervalls und erlaubt es der \lstinline$for$-Schleife acht Iterationen durchzuf�hren. Die berechneten Zahlen werden mit Hilfe der \lstinline$append$-Methode der List hinzugef�gt. 
	\lstinputlisting[language=Python, firstline=16,lastline=20]{exercises/src/Collections/CollectionsAufgabe2List.py}
	
\end{enumerate}