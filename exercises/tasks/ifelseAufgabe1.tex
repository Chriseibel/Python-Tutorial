\begin{enumerate}
  
\item Ein Programm soll eine Variable �berpr�fen. Hat die Variable den Wert '1' soll sie 'eins' ausgeben. Bei dem Wert '2' soll sie 'zwei' ausgeben, andernfalls soll sie 'weder eins, noch zwei' ausgeben.

\item Von zwei int-Variablen soll die gr��ere von beiden bestimmt werden. L�sen Sie dies mithilfe eines Programms.

Beispiel: Bei den Werten a=5 und b=2 soll ausgegeben werden: 'a ist gr��er als b'.

\item L�sen Sie Aufgabe b) als Conditional Expression. (Der Fall, dass a und b gleich sind muss hier nicht beachtet werden)

\item �ndern Sie Ihr Programm aus Aufgabe b) so ab, dass das Maximum von drei Zahlen statt von zwei bestimmt werden kann.

\item Der Benutzer einer Applikation will herausfinden, f�r welche Fahrzeuge er bereits einen F�hrerschein machen darf. Dabei gilt: Ab 15 Jahren Mofa, ab 16 Jahren Motorroller und kleinere Motorr�der, ab 17 Jahren begleitetes Fahren, ab 18 Jahren PKW (alleine), ab 21 Jahren LKW und ab 25 Jahren jegliche Motorr�der. 

Entwerfen Sie ein Programm, dass zu dem Alter des Benutzers alle m�glichen Fahrzeuge ausgibt.
\end{enumerate}