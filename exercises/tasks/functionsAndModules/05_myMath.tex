% !TeX root = ../../pythonTutorial.tex
Erstellen Sie ein eigenes Modul mit den bereits erstellten Methoden aus dieser Übung. Erweitern Sie das Modul um die Funktionen \emph{Summe} und \emph{Produkt}. Beide akzeptieren eine unbestimmte Anzahl von Parametern. 

\begin{lstlisting}[language=Python]
# Neue mathematische Funktionen

def sum(...):
	... # addiert alle Zahlen zu einer Summe
	
def product(...):
	... # multipliziert alle Zahlen zu einem Produkt
\end{lstlisting}

Speichern Sie das Modul ab und nutzen Sie es in einem anderen Programm. Kopieren Sie den \emph{print}-Befehl und vergleichen Sie das Ergebnis.

\begin{lstlisting}[language=Python]
# Nutzen des eigenen Moduls

import myModule

print(myModule.sum(
	myModule.product(2,3), myModule.product(3,5)))
# Ausgabe: 21
print(myModule.product(
	myModule.sum(2,3,4,5), myModule.product(0,9)))
# Ausgabe: 0

\end{lstlisting}
