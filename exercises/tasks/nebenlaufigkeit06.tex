\label{nebenlaufigkeit06}
Erweitern Sie das \lstinline$Counter$ und \lstinline$IncrementerThread$ Beispiel um Condition Variablen.
\begin{enumerate}
\item �bernehmen Sie die �nderungen aus der angepassten
\hyperref[threads:lst:counter_condition_variable_example]{\lstinline$Counter$-Klasse} in Ihre eigene.

\item Stellen Sie sicher, dass der richtige Thread aufgeweckt wird, sobald der geteilte Zustand
ver�ndert wird.

\item Erweitern Sie \lstinline$IncrementThread$ um ein neues Attribut \lstinline$digit$.
Dem Konstruktor soll per Parameter ein Wert f�r das neue Attribut �bergeben werden.

\item Erg�nzen Sie in \lstinline$IncrementThread$ eine \lstinline$check_condition()$-Methode, die
\lstinline$True$ zur�ck gibt, wenn die letzte Ziffer des \lstinline$Counters$ dem neuen Attribut
\lstinline$digit$ entspricht.

\item Passen Sie die \lstinline$run()$-Methode des \lstinline$IncrementThreads$ so an, dass 
\lstinline$increment()$ nur dann aufgerufen wird, wenn seine Bedingung erf�llt ist.

\tip{
Sie erhalten das \lstinline$Condition$-Objekt des \lstinline$Counters$ innerhalb der
\lstinline$IncrementerThread$-Klasse mit folgendem Aufruf:
\lstinline$self.counter.cv$
}

\item Erg�nzen Sie das Programm mit passenden Ausgaben, um nachzuvollziehen, welcher Thread 
aufgeweckt wurde und gerade inkrementiert.

\end{enumerate}

\warning{
Achten Sie darauf, dass Sie Ihren \lstinline$Counter$ nur um 1 erh�hen!
}
