Erzeugen Sie ein \lstinline!numpy!-Array mit den Werten
\lstinline![!%
\(-2\pi\), %
\(-1.5\pi\), %
\(-\pi\), %
\(-0.5\pi\), %
\(0\), %
\(0.5\pi\), %
\(\pi\), %
\(1.5\pi\), %
\(2\pi\)\lstinline!]!%
. F�hren Sie
die nachfolgenden Berechnungen darauf aus und geben Sie das jeweilige Ergebnis
mit \lstinline!print()! auf der Konsole aus.
\begin{enumerate}
	\item Berechnen Sie den Sinus, Cosinus und Tangens des Arrays.
	\item Multiplizieren Sie das Array elementweise mit sich selbst.
	\item Addieren Sie die eulersche Zahl zu jedem Wert im Array.
	\item Wandeln Sie das Array von Radians in Grad um.
	\item Runden Sie s�mtliche Werte im Array auf zwei Nachkommastellen und
	      bilden Sie die Summe des Ergebisses. Welchen Wert erwarten Sie f�r diese
				Summe? Unterscheidet sich das Ergebnis von der Summe der Werte im
				urspr�nglichen Array?
\end{enumerate}
