% !TeX root = ../pythonTutorial.tex

\section{Kontrollstrukturen}

Die Kontrollstrukturen in Python haben einen formalen Unterschied zu Java oder C++, funktional sind sie identisch. In Python werden keine geschweiften Klammern genutzt, um die Bl�cke der einzelnen Abfragen abzugrenzen. Dazu gen�gt das Einr�cken der Anweisung. Dies gilt sowohl f�r if-then-else-Bedingungen und Conditional Expressions, als auch f�r Schleifen. Im Folgenden schauen wir uns die einzelnen Strukturen im Detail und mit Beispielen an.

\subsection{If-then-else}

Die if-then-else-Struktur erm�glicht es - wie wir es bereits kennen - simple wenn-dann Abfragen zu t�tigen.

Mehrere Abfragem�glichkeiten werden mit elif markiert. \ref{refif}

\begin{lstlisting}[caption={If-then-else},label=refif]
if statement1:
	print("Fall 1")
elif statement2:
	print("Fall 2")
else:
	print("Fall 3")

\end{lstlisting}

\textbf{Conditional Expressions}

Die Conditional Expressions (engl. bedingte Ausdr�cke) stellen eine kompaktere Schreibweise als if-then-else-Bedinungen dar. \ref{refcond}

\begin{lstlisting}[caption={Conditional Expressions},label=refcond]
// Klassisches If-Else
if wort == "start":
	x = "los"
else:
	x = halt"
	
// If-Else als Conditional Expression
x = ("los" if wort == "start" else "halt")

\end{lstlisting}

\newpage
\subsection{Schleifen}

Python hat sowohl while- \ref{refwhile} als auch for-Schleifen \ref{reffor}, welche wir uns beide im Folgenden genauer ansehen werden. Schleifen bestehen aus einer Anweisung und einem Kontrollblock, welcher solange durchlaufen wird, bis die Anweisung oder ein Abbruchkriterium erf�llt wurde. Schleifen die niemals ein Abbruchkriterium erf�llen und so endlos durchlaufen werden hei�en Endlosschleifen. Diese f�hren dazu, dass der Interpreter irgendwann den Geist aufgibt und abbricht.

\begin{lstlisting}[caption={While-Schleife},label=refwhile]
// While Schleife
while Bedingung:
	Anweisungsblock
	if Bedingung:
		Anweisungsblock
		continue
	if Bedingung:
		Anweisungsblock
		break
	Anweisungsblock

\end{lstlisting}

\begin{lstlisting}[caption={For-Schleife},label=reffor]
// For Schleife
for Variable in Objekt:
	Anweisungsblock
	if Bedingung:
		Anweisungsblock
		continue
	Anweisungsblock
	if Bedingung:
		Anweisungsblock
		break
	Anweisungsblock

\end{lstlisting}

\newpage
\textbf{Enums}

Enums dienen in den objektorientierten Programmiersprachen zur Aufz�hlung von Ausdr�cken einer endlichen Menge. So werden zum Beispiel Jahreszeiten, Monate oder Farben oft als Enums umgesetzt. \ref{refenum}

\begin{lstlisting}[caption={Enums},label=refenum]
from enum import Enum
class Color(Enum):
	RED = 1
	GREEN = 2
	BLUE = 3

\end{lstlisting}




