\section{NumPy}

NumPy ist eine Python-Bibliothek f�r wissenschaftliches Rechnen.
Sie beinhaltet unter anderem Folgendes:

\begin{itemize}
  \item m�chtige $n$-dimensionale Array-Objekte
  \item Werkzeuge zur Integration von C und Fortran
  \item Lineare Algebra, Fouriertransformation, Erzeugung von Zufallszahlen
\end{itemize}

% Zitat: http://www.numpy.org

Um NumPy zu installieren, kann der Befehl \lstinline!pip install numpy!
verwendet werden.

\subsection{Arrays}

Der Array-Datentyp von NumPy hei�t \lstinline!numpy.ndarray!. Anders als Pythons
Listentyp \lstinline!list! unterst�tzt \lstinline!numpy.ndarray! numerische
Operationen auf Arrays. Die Grundrechenarten werden zwischen zwei Arrays
elementweise und zwischen Array und \lstinline!int!/\lstinline!float! f�r alle
Elemente des Arrays durchgef�hrt.

So kann etwa jeder Wert in einem Array mit den folgenden Anweisungen um drei
erh�ht werden:
\begin{lstlisting}
import numpy as np
a = np.array([1,2,3])
a + 3 # [4 5 6]
\end{lstlisting}

Subtraktion, Multiplikation, Division, Ganzzahldivision und Potenzieren
funktionieren analog:\footnote{Der Import von \lstinline!numpy! wird der
�bersichtlichkeit halber nachfolgend ausgelassen.}
\begin{lstlisting}
a = np.array([1,2,3])
a - 3  # [-2 -1  0]
a * 3  # [3 6 9]
a / 3  # [0.33333333 0.66666667 1.        ]
a // 3 # [0 0 1]
a ** 3 # [ 1  8 27]
\end{lstlisting}

Zwei Arrays gleicher L�nge k�nnen elementweise miteinander verkn�pft werden:
\begin{lstlisting}
a = np.array([1,2,3])
b = np.array([4,5,6])
a + b  # [5 7 9]
a - b  # [-3 -3 -3]
a * b  # [ 4 10 18]
a / b  # [0.25 0.4  0.5 ]
a ** b # [  1  32 729]
a // b # [0 0 0]
\end{lstlisting}

Um ein NumPy-Array zu erzeugen, wird ein \lstinline!list!-Objekt an
\lstinline!np.array()! �bergeben, dabei wird der in der \lstinline{list}
enthaltene Datentyp in einem Datentyp von NumPy konvertiert. Um den Datentyp
eines Arrays herauszufinden, wird \lstinline!.dtype.name! genutzt. Anders
als bei \lstinline!list! m�ssen s�mtliche Elemente eines Arrays den gleichen
Typ haben.
\begin{lstlisting}
a = np.array([1,2,3])
a.dtype.name # 'int64'
b = np.array([1.4,2.5,3.6])
a.dtype.name # 'float64'
\end{lstlisting}

Wenn \lstinline!int! und \lstinline!float! gemischt �bergeben werden,
konvertiert NumPy in den Flie�kommatyp. Wie viele Bit f�r einen Integer bzw. ein
Float zur Verf�gung stehen, ist von der Prozessorarchitektur abh�ngig. Moderne
Computer unterst�zten in der Regel eine Gr��e von 64 Bit.


\subsection{Konstanten und Funktionen}
Es stehen f�r die mathematische Anwendungen auch Konstanten zur Verf�gung,
darunter die Folgenden mit den entsprechenden Werten und Pr�zisionen mit
\lstinline!float64!:
\begin{lstlisting}
>>> np.pi
3.141592653589793
>>> np.e
2.718281828459045
>>> np.euler_gamma
0.5772156649015329
>>> np.PINF
inf
>>> np.NINF
-inf
>>> np.NAN
nan
>>> np.PZERO
0.0
>>> np.NZERO
-0.0
>>> np.NAN
nan
\end{lstlisting}

\lstinline!np.NZERO! steht f�r die negative Darstellung der Null bei
Flie�kommazahlen, \lstinline!np.PZERO! f�r die positive Darstellung.

NumPy unterst�tzt eine Vielzahl an mathematischen Funktionen, darunter unter
anderem trigonometrische Funktionen, Rundungs-,
Summations- und Multiplikationsfunktionen und Funktionen zur Behandlung
komplexer Zahlen.

Die grundlegenden trigonometrischen Funktionen sind selbsterkl�rend, sie werden
elementweise auf das Array angewendet:
\begin{lstlisting}
>>> a = np.array([0, np.pi/6, np.pi/4, np.pi/3, np.pi/2])
>>> np.sin(a)
[0.         0.5        0.70710678 0.8660254  1.        ]
>>> np.cos(a)
[1.00000000e+00 8.66025404e-01 7.07106781e-01 5.00000000e-01
6.12323400e-17]
>> np.tan(a)
[0.00000000e+00, 5.77350269e-01, 1.00000000e+00,
1.73205081e+00, 1.63312394e+16]
\end{lstlisting}

Auch die Umrechung von Radians in Grad
\begin{lstlisting}
>>> a = np.array([0, np.pi/6, np.pi/4, np.pi/3, np.pi/2])
>>> np.degrees(a)
[ 0., 30., 45., 60., 90.]
\end{lstlisting}
und Grad in Radians m�glich:
\begin{lstlisting}
>>> a = np.array([ 0, 30, 45, 60, 90])
>>> np.radians(a)
[0.         0.52359878 0.78539816 1.04719755 1.57079633]
\end{lstlisting}

Mit \lstinline!np.around! k�nnen s�mtliche Werte im Array auf eine bestimmte
Anzahl von Stellen gerundet werden. Ohne Angabe eines zweiten Arguments wird
kaufm�nnisch auf die n�chste Ganzzahl gerundet.
\begin{lstlisting}
>>> a = np.array([1.49, 1.5, 1.51])
>>> np.round(a)
[1. 2. 2.]
\end{lstlisting}
Mit dem optionalen zweiten Argument wird die Anzahl an Nachkommastellen, auf die
gerundet werden soll, angegeben:
\begin{lstlisting}
>>> a = np.array([1.25, 1.53, 1.99])
>>> np.round(a, 1)
[1.2, 1.5, 2. ]
\end{lstlisting}

Um alle Elemente eines Arrays aufzusummieren, wird die Funktion \\
\lstinline!np.sum()! verwendet.
\begin{lstlisting}
>>> a = np.array([1, 2, 3])
>>> np.sum(a)
6
\end{lstlisting}
Mit \lstinline!np.prod()! k�nnen die Elemente der Liste miteinander
multipliziert werden.
\begin{lstlisting}
>>> a = np.array([2, 3, 4])
>>> np.prod(a)
24
\end{lstlisting}


Sollten \lstinline!nan! (not a number) im Array vorkommen k�nnen, so kann
\lstinline!np.nansum! beziehungsweise \lstinline!np.nanprod! verwendet werden.
Bei \lstinline!np.nansum! werden \lstinline!nan! als \lstinline!0!
interpretiert,
\begin{lstlisting}
>>> a = np.array([np.NAN, 1, 2, 3])
>>> np.sum(a)
nan
>>> np.nansum(a)
6.0
\end{lstlisting}
bei \lstinline!np.nanprod! als \lstinline!1!:
\begin{lstlisting}
>>> a = np.array([np.NAN, 2, 3, 4])
>>> np.prod(a)
nan
>>> np.nanprod(a)
24.0
\end{lstlisting}
Das Ergebnis der Addition bzw. Multiplikation ist vom Typ \lstinline!float64!.
\lstinline!nan! ist ein valider Flie�zahlwert, daher werden die restlichen Werte
in der zu konvertierenden \lstinline!list! von \lstinline!int! zu
\lstinline!float64! umgewandelt.
