% !TeX root = ../../pythonTutorial.tex

\section{Elementare Datentypen}

�hnlich wie bei Java und C oder C++, gibt es auch in Python Variablen. Allerdings gibt es dabei immense Unterschiede zu den anderen Programmiersprachen, weshalb sich ein genauerer Blick auf die einzelnen Datentypen in jedem Fall lohnt. Bei vielen bekannten Sprachen wird einer Variablen ein bestimmter Datentyp zugeordnet (deklariert). Der Datentyp kann darauf folgend zur Laufzeit nicht wieder ge�ndert werden, der Wert innerhalb des Datentyps allerdings schon. So lassen sich in eine Variable des Typ Integer beispielsweise keine String-Werte speichern. In Python hingegen, ist dies ohne weiteres m�glich. Hier wird n�mlich g�nzlich auf eine explizite Typdeklaration verzichtet. Zeigt eine Variable beispielsweise auf eine ganze Zahl, so wird diese als ein Objekt vom Typ Integer interpretiert. Allerdings kann man sie im n�chsten Schritt einfach auf ein String-Objekt zeigen lassen. 
Soviel zum allgemeinen Unterschied zu den anderen Programmiersprachen. Betrachten wir nun die Datentypen etwas genauer.

\newpage
\subsection{Zahlenoperatoren}

Da in Python auf Typdeklarationen verzichtet wird, muss dieser nicht beim Anlegen der Variable ber�cksichtigt werden. Wird eine ganze Zahl (Integer)ben�tigt, kann diese falls n�tig auch in eine Gleitkommazahl (float) umgewandelt werden, ohne gro� etwas am Code zu �ndern. Python deklariert im Hintergrund selbst und spart so unn�tige Komplexit�ten und Fehlerquellen (Beispiel in \ref{basicDatatypes:lst:refzahl}).


\begin{lstlisting}[caption={Zahlenoperatoren},label=basicDatatypes:lst:refzahl]
i = 42
type(i)
// Ausgabe: <class 'int'>
i = 42.22
type(i)
// Ausgabe: <class 'float'>
\end{lstlisting}

\textbf{Boolean}

Boolean gibt an, ob ein Statement true oder false ist. Dadurch lassen sich Fallunterscheidungen oder Abfragen erm�glichen (Beispiel in Listing \ref{basicDatatypes:lst:refbool}). 

\begin{lstlisting}[caption={Boolean},label=basicDatatypes:lst:refbool]
i = True
i
// Ausgabe: True

\end{lstlisting}

\textbf{String}

Der String ist eine Zeichenkette, also eine Aneinanderreihung von verschiedenen Zeichen. Dazu z�hlen W�rter, aber auch beispielsweise Hexadezimal-Codes oder E-Mail Adressen.

Wie in den meisten objektorientierten Programmiersprachen, lassen sich auch in Python die einzelnen Zeichen eines Strings abrufen in dem der dazugeh�rige Index abgerufen wird.

Wie in Listing \ref{basicDatatypes:lst:refstring} kann die L�nge des gesamten Strings durch einfache Abfrage angezeigt werden. 

\begin{lstlisting}[caption={Strings},label=basicDatatypes:lst:refstring]
i = "Python"
print (i)
// Ausgabe: Python

print(i[0])
// Ausgabe: P

print(len(i))
// Ausgabe: 6

\end{lstlisting}