% !TeX root = ../../pythonTutorial.tex
\section{Threads}

\label{threads}

In Python werden zwei APIs zur Verwendung von Threads angeboten, die Low-Level
API aus dem \_thread-Modul und die Higher-Level API aud dem threading-Modul.
Es wird sich an dieser Stelle auf das threading-Modul beschr�nkt, da es Intern auf
dem \_thread-Modul basiert und eine Schnittstelle anbietet, welche das Programmieren
von Multithreaded Applikationen erleichert.
Diese Schnittstelle ist an der Thread-Schnittstelle aus Java angelehnt und sollte daher f�r
Java-Entwickler leicht zu verwenden sein.
Allerdings gibt es einige Unterschiede zwischen dem Python-Modul und der Java-Implementierung.
So sind Bedingungsvariablen und Locks seperate Objekte in Python und es ist auch
nur eine Teilmenge des Verhaltens eine Java-Threads in Python verf�gbar.
Ein Python-Thread kennt keine Priorit�ten und Thread-Gruppen und er kann nicht zerst�rt,
gestopped, angehalten, fortgesetzt oder unterbrochen werden.
Soweit vorhanden sind die statischen Methoden aus der Java-Thread-Klasse auf Modul-Ebene
in Python implementiert.

\subsection{Thread Objekte}

Es gibt zwei M�glichkeiten einen Thread zu erzeugen. Entweder wird dem Konstruktor ein
aufrufbares Objekt �bergeben,
\begin{lstlisting}
import threading

def aufgabe():
    # Nebenl�ufig ausgef�hrte Aufgabe

thread = threading.Thread(target=aufgabe)
thread.start()
\end{lstlisting}
oder die \lstinline$run()$-Methode wird in einer von \lstinline$Thread$ abgeleiteten Klasse �berschrieben.
\begin{lstlisting}
import threading

class MeinThread(threading.Thread):
    def run(self):
        # Nebenl�ufig ausgef�hrte Aufgabe

thread = MeinThread()
thread.start()
\end{lstlisting}

Der Konstruktor  der \lstinline$Thread$-Klasse bietet noch weitere Parameter an:
\begin{itemize}
    \item \lstinline$group$ sollte immer \lstinline$None$ sein.
    Es ist aktuell reserviert f�r sp�tere Erweiterungen.
    \item \lstinline$name$ setzt den Namen des Threads.
    \item \lstinline$args$ ist ein Tupel aus Parameter f�r das mit \lstinline$target$ definiert
    aufrufbare Objekt.
    \item \lstinline$kwargs$ ist ein Dictionary aus Schl�sselwort-Parameter f�r \lstinline$target$.
    \item \lstinline$deamon$ setzt die D�mon-Eigenschaft des Threads.
\end{itemize}

Es ist anzumerken, dass ein \lstinline$Thread$-Objekts bei seiner Erzeugung noch nicht gestartet wird.
Hierzu muss explizit die \lstinline$start()$-Methode aufgerufen werden.
Wurde ein Thread gestartet wird er als ''lebendig'' angesehen. 
Dies bleibt er solange, bis seine \lstinline$run()$-Methode verlassen wurde.
Hierbei macht es keinen Unterschied, ob sie regul�r verlassen wurde oder Aufgrund einer Exception.
Der aktuelle Status eines Threads kann mittels der \lstinline$is_alive()$-Methode abgefragt werden.
Soll auf das Ende eines anderen Threads gewartet werden, so kann seine
\lstinline$join()$-Methode aufgerufen werden. 
Hiermit wird der aufrufende Thread blockiert, bis der andere beendet ist.
Die \lstinline$join$-Methode nimmt einen optionalen Parameter des Typen \lstinline$float$ entgegen,
der als Timeout in Sekunden dient.
Wird eine Timeout angegeben, ist es wichtig, dass nach dem \lstinline$join()$-Aufruf die Methode 
\lstinline$is_alive()$ aufzurufen.
Da \lstinline$join()$ immer \lstinline$None$ zur�ckgibt, ist es ansonsten nicht m�glich zu wissen, ob
der Thread tats�chlich beendet wurde, oder nur der Timeout abgelaufen ist.

\kontrollfrage{
\item[\kontroll] Wie kann auf das Ende der Ausf�hrung eines Threads gewartet werden?
}

Jeder Thread besitzt einen Namen, welcher initial �ber den Konstruktor oder direkt �ber das
\lstinline$name$-Attribut gesetzt werden kann. 
Threads k�nnen als D�mon gekennzeichnet werden.
Sobald nur noch D�mon-Threads aktiv sind, wird das Python-Programm beendet.
Die D�mon-Eigenschaft kann initial �ber den Konstruktor gesetzt werden und �bernimmt
den Werte des erzeugenden Threads als Defaultwert.
�ber das \lstinline$deamon$-Attribut eines Threads kann die Eigenschaft abgefragt und gesetzt werden.
Hierbei ist es wichtig, dass die Eigenschaft immer vor dem Aufruf der \lstinline$start()$-Methode
gesetzt wird.
Wird sie nach dem Starten des Threads ge�ndert, so wird ein \lstinline$RuntimeError$ geworfen.

\warning{
	D�mon-Threads werden sofort beendet, wenn keine normalen Threads mehr aktiv sind.
	Das hei�t, dass ihre Ressourcen wie zum Beispiel ge�ffnete Dateien oder Datenbanktransaktionen
	gegebenenfalls nicht ordentlich freigegeben werden.
	Um dies zu verhindern sollten die Threads nicht die D�mon-Eigenschaft besitzen und es sollten 
	geeignete Signalisierungsmechanismen eingesetzt werden (siehe \lstinline$Event$-Objekte). %ggf. entfernen falls Kapitel nicht in finaler Abgabe
}

\uebung
\aufgabe{nebenlaufigkeit01}
\aufgabe{nebenlaufigkeit02}
