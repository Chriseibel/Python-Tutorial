\include{lstlisting}
\section{Grundlagen}
Python ist eine Scriptsprache die 
scriptsprache
anwendungszweck

\subsection{Installation des Interpreters}
Python kann auf der Webseite https://www.python.org bezogen werden.
Im folgenden wird die Python in der Version 3 behandelt.





\subsection{Syntax}
Folgende syntaktische Besonderheiten Bringt Python mit:
\subsubsection{Leerzeichen und Einrückung}
Um in Python Blöcke auszuzeichnen können nicht wie in Java und C++ geschweifte Klammern genutzt werden. 
In Python ist hierfür entweder der Tabulator oder 4 aufeinander folgende Leerzeichen vorgesehen.
Somit kann bei zum Beispiel einer IF-Abfrage der nachfolgende Block leicht falsch zugeordnet werden wenn in den nachfolgenden Zeilen die Einrückung übersehen wird.
\subsubsection{Kommentare}
Innerhalb Python wird zwischen Zeilen und Blockkommentaren unterschieden.
Blockkommentare werden über das Rautensymbol (\# ) eingeleitet.
Blockkommentare über drei aufeinander folgenden Anführungzeichen ('''''').
Im folgenden jeweils ein Beispiel.
\lstinputlisting{chapters/sections/listings/comment.py}
\subsubsection{Typsicherheit}
Anders als bei Java und C++ ist Python eine nur schwach typisierte Sprache.
Somit ist bei der Initialisierung keine Typangabe erforderlich. 
Der Datentyp wird beim Initialisieren dynamisch ermittelt und automatisch zugewiesen.
\subsubsection{prozedurale Programmierung}
\subsection{Interpreter}
übersetz das ihm übergebene Script
kann als Umgebung genutzt werden


\subsection{Beispiel \glqq Hallo World!\grqq}
Hier ein einfaches \glqq Hallo World!\grqq -Beispiel.

\lstinputlisting{chapters/sections/listings/helloworld.py}
