% !TeX root = ../../pythonTutorial.tex
\section{JSON}
\label{filehandling:section:json}

JavaScript Object Notation (JSON) ist ein Format f�r den Austausch von Daten, das unabh�ngig von der Programmiersprache ist. Aufgrund von Konventionen,
die dieses Format mit Programmiersprachen aus der C-Familie, wie C, C++, Java oder Python teilt, liefert es eine Programmierern bekannte Textstruktur.

In Python 3 ist nativ das json-Package enthalten, das das Arbeiten mit
dem JSON-Format erm�glicht. Mithilfe des folgenden Codes binden wir das Package in das Projekt ein. 
\lstinputlisting[language=Python, linerange={1-1,3-5}]{chapters/filehandling/src/json_in_python/JsonInPython.py}
\label{filehandling:lst:importpackage}

Ein gegebener JSON-String wird �ber die \lstinline$loads()$-Methode in ein in Python existierendes, entsprechendes Objekt geparst.
In diesem Fall wird ein Dictionary angelegt.
\randnotiz{JSON zu Python}
\lstinputlisting[language=Python, linerange={1-1,7-19}]{chapters/filehandling/src/json_in_python/JsonInPython.py}
\label{filehandling:lst:loads}

F�r das Umwandeln eines Python-Objekts in einen JSON-String verwenden wir die 
\lstinline$dumps()$-Methode.
\randnotiz{Python zu JSON}
\lstinputlisting[language=Python, linerange={1-1,21-35}]{chapters/filehandling/src/json_in_python/JsonInPython.py}
\label{filehandling:lst:dumps}

Konvertieren wir Python- zu JSON-Objekte, werden diese im\\
JSON-�quivalent (JavaScript) angelegt.

Wenn wir einen Dictionary mit mehreren Schl�ssel-Objekt-Paaren anlegen,
werden wir bei der Ausgabe des JSON-Objekts feststellen, dass diese auf eine Zeile beschr�nkt sind. 
\lstinputlisting[language=Python, linerange={1-1,37-55}]{chapters/filehandling/src/json_in_python/JsonInPython.py}
\label{filehandling:lst:format1}

Zur Formatierung unserer Ausgabe verwenden wir die \lstinline$dumps()$-Methode.
Mithilfe des \lstinline$indent$-Parameters k�nnen wir festlegen, ob und wie weit die Textstruktur einger�ckt werden soll. 
Der \lstinline$separators$-Parameter legt die\\
Trennzeichen fest und mit \lstinline$sort_keys=True$ wird die Ausgabe der Schl�ssel lexikografisch sortiert.
\lstinputlisting[language=Python, linerange={1-1,37-37,56-81}]{chapters/filehandling/src/json_in_python/JsonInPython.py}  
\label{filehandling:lst:format2}

\subsection{Zusammenfassung}
\label{filehandling:subsection:zusammenfassungdateienlesenundschreiben}

In diesem Kapitel haben wir uns mit dem lesen und beschreiben einer Datei auseinandergesetzt.
Dies geschieht in Python mithilfe eines \lstinline$fileObject$ um eine Datei zu erstellen, �ndern, l�schen und abzuspeichern.
Dabei kann eine Datei als Textdatei oder Bin�rdatei interpretiert werden. 
Eine der wichtigsten Methoden stellt hierbei die \lstinline$open()$-Methode dar. 
Diese erm�glicht uns das Erstellen, �ffnen, Aktualisieren, Lesen und Beschreiben einer Datei.
Dabei ist zu beachten, dass die Datei direkt nach der Ausf�hrung der gew�nschten Operationen mithilfe der \lstinline$close()$-Methode geschlossen wird.
Um dies nicht zu vergessen, besteht in Python auch die M�glichkeit ein automatisches Schlie�en mit dem \lstinline$With$-Statement zu erwirken. 
Abschlie�end haben wir noch den Zugriff auf die wichtigsten Attribute, die ein \lstinline$fileObject$ besitzt, kennengelernt und uns mit dem Standardisierten Datenaustausch mittels JSON besch�ftigt.

