% !TeX root = ../../pythonTutorial.tex
\section{Fehler- und Ausnahmebehandlung}
\label{filehandling:section:fehlerundausnahmebehandlung}

In diesem Kapitel besch�ftigen wir uns mit der Fehler- und Ausnahmebehandlung in Python.
Dabei sollte darauf geachtet werden, dass �berall da, wo ein potenzieller Fehler auftreten kann, die Ma�nahmen zur Fehler- und Ausnahmebehandlung angewendet werden.
Somit k�nnen wir entsprechende Syntaxfehler zur Laufzeit abfangen und geeignet behandeln, ohne das unser Programm vorzeitig durch einen Absturz beendet wird. 

\subsection{M�gliche Fehlerquellen}
\label{filehandling:section:fehlerquellen}
Die M�glichkeiten der Fehlerquellen sind vielseitig. Eine der Bekanntesten davon ist wohl ein Eingabefehler durch den Benutzer. 
Dieser soll beispielsweise eine Zahl mithilfe der Tastatur eingeben, damit diese durch das Programm weiterverarbeitet werden kann. 
Durch ein Vertippen des Anwenders wird ein Text in Form eines einzelnen Buchstabens anstatt einer Zahl �bergeben.
Dies f�hrt zur Laufzeit zu einem Fehler, das Programm hat eine Zahl an Stelle eines Textes erwartet.
Eine andere Fehlerquelle w�re die Division durch die Zahl Null.
Ebenso k�nnte der Zugriffsversuch auf eine Datei zum Bearbeiten fehlschlagen, da diese zu dem aktuellen Zeitpunkt noch nicht existiert.
\\ \\
Aus diesen genannten Ursachen ist eine Fehler- und Ausnahmebehandlung sinnvoll und sollte von einem Programmierer f�r einen m�glichen Einsatz stets bedacht werden.

\subsection{try, except, else und finally}
\label{filehandling:section:tryblock}


Der \lstinline$try-Block$  \randnotiz{try}wird mit dem Schl�sselwort \lstinline$try$ gefolgt von einem Doppelpunkt eingeleitet. 
Anschlie�end wird der auszuf�hrende Code, der einen Fehler beinhalten k�nnte, darin angegeben.
Mit dem Schl�sselwort\randnotiz{except} \lstinline$except$   und dem Namen der zu behandelnden Fehlerklasse wie beispielsweise \\ \lstinline$ZeroDivisionError$ kann ein entsprechender Fehler geeignet behandelt werden. 
Dabei sind auch mehrere \lstinline$except$-Anweisungen mit unterschiedlichen Fehlerklassen m�glich, um eine entsprechende Behandlung zu erm�glichen. 
Somit kann jeder Fehler nach seiner eigenen Art und Weise nach dem Auftreten konsequent und individuell behandelt werden. 
Auch die Erstellung von eigenen Fehlerklassen ist in Python m�glich, dazu sp�ter mehr.
\\ \\
Einige wichtige und g�ngige Fehlerklassen sind hierbei: 
\\ \\
\lstinline$"ZeroDivisonError"$: Tritt auf bei einer Division durch die Zahl null.

\lstinline$"FileNotFoundError"$: Tritt auf, wenn die zu �ffnende Datei nicht gefunden werden kann.

\lstinline$"IOError"$: Tritt auf, wenn man auf eine Ressource zugreifen m�chte, die momentan nicht verf�gbar ist. Beispielsweise der Zugriff auf einen Drucker, der zu dem aktuellen Zeitpunkt sich in dem Status \glqq{}offline\grqq{} befindet.

\lstinline$"ValueError"$: Tritt auf, wenn ein anderer Datentyp als der erwartete Verarbeiteten werden soll. 
Beispiel: Es wird eine Zahl erwartet, aber ein Text �bergeben.

Dar�ber hinaus gibt es noch weitere wie z. B. ImportError, KeyError, MemoryError, NameError, TypeError und viele mehr, die hier im Kontext nicht weiter erl�utert werden.

Im nachfolgenden Listing wird ein Fehler provoziert, indem wir eine Division mit der Zahl Null herbeif�hren. 
Hierbei wird die Fehlerbehandlung mit der Klasse \lstinline$ZeroDivisonError$ abgefangen und anschlie�end das Programm durch die Fehlerbehandlung ordnungsgem�� beendet. 
Dabei wird die nachfolgende \lstinline$else-klausel$ durch das Auftreten und Abfangen des Fehlers nicht ausgef�hrt.

\lstinputlisting[language=Python,
 firstline=1,lastline=16]{chapters/filehandling/src/fehler_und_ausnahmebehandlung/ExceptionHandling.py}
\label{filehandling:lst:zerodivisonerror}

Die \lstinline$else$-Anweisung \randnotiz{else}kann im Code optional mit angegeben werden und wird nur ausgef�hrt, 
falls es zu keiner Ausnahme in dem \lstinline$try-Block$ kommt.
Somit wird nach der Ausf�hrung des \lstinline$try-Blocks$ auch der in \lstinline$else$ stehende Code ausgef�hrt, wie es das nachfolgende Listing demonstriert:

\lstinputlisting[language=Python, linerange={1-3,20-32}]{chapters/filehandling/src/fehler_und_ausnahmebehandlung/ExceptionHandling.py}
\label{filehandling:lst:else}
