% !TeX root = ../../pythonTutorial.tex
\chapter{Fehler- und Ausnahmebehandlung}
\label{filehandling:section:fehlerundausnahmebehandlung}

In diesem Kapitel besch�ftigen wir uns mit der Fehler- und Ausnahmebehandlung in Python.
Dabei sollte darauf geachtet werden, dass �berall wo ein potenzieller Fehler auftreten kann, die Ma�nahmen zur Fehler- und Ausnahmebehandlung angewendet werden.
Somit k�nnen wir entsprechende Syntaxfehler zur Laufzeit abfangen und geeignet behandeln, ohne das unser Programm vorzeitig durch einen Absturz beendet wird. 

\section{M�gliche Fehlerquellen}
\label{filehandling:section:fehlerquellen}
Die M�glichkeiten der Fehlerquellen sind vielseitig. Eine der Bekanntesten davon ist wohl ein Eingabefehler durch den Benutzer. Der Benutzer soll beispielsweise eine Zahl mithilfe der Tastatur eingeben, damit diese durch das Programm weiterverarbeitet werden kann. Durch ein Vertippen des Anwenders wird ein Text in Form eines einzelnen Buchstabens anstatt einer Zahl �bergeben.
Dies f�hrt zur Laufzeit zu einem Fehler.
An dieser Stelle hat n�mlich das Programm eine Zahl erwartet und kann mit dem eingegebenen Text, nichts anfangen. 
Eine andere Fehlerquelle w�re die Division mit der Zahl null, die einen Fehler herbeif�hrt.
Ebenso k�nnte der Zugriffsversuch auf eine Datei zum Bearbeiten fehlschlagen, da diese zu dem aktuellen Zeitpunkt noch nicht existiert.
\\ \\
Aus diesen genannten Ursachen ist eine Fehler- und Ausnahmebehandlung sinnvoll und sollte von einem Programmierer f�r einen m�glichen Einsatz stets bedacht werden.

