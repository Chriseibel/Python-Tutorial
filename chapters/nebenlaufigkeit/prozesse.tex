% !TeX root = ../../pythonTutorial.tex
\subsection{Prozesse}
\label{prozesse:subsection:prozesse}

Die bisher betrachteten Beispiele und Aufgabe waren nicht CPU-gebunden, weshalb die Auswirkungen des
in Abschnit \ref{parallelit�t_in_python:subsection:parallelit�t_in_python} angesprochenen GILs nicht
ersichtlich wurden. 
Wird nun aber das folgende Beispiel betrachtet, in dem die Anzahl der Primzahlen im Zahlenbereich 1
bis 10 Millionen gesucht wird, ist dies nicht mehr der Fall:

\label{prozesse:lst:primzahlen_threads}
\lstinputlisting[language=Python]{chapters/nebenlaufigkeit/src/primzahlen_threads.py}

Jeder Thread bekommt hierbei einen Bereich aus 1 Milliionen Zahlen.
Es wird auch die Zeit gemessen, die die Threads ben�tigen, um die Primzahlen zu z�hlen.
Wird diese Programm ausgef�hrt, kann eine m�gliche Ausgabe wie folgt aussehen:
\label{prozesse:lst:thread_ausgabe}
\begin{lstlisting}
Found  67883  primes
Found  78498  primes
Found  70435  primes
Found  63799  primes
Found  65367  primes
Found  64336  primes
Found  66330  primes
Found  62712  primes
Found  62090  primes
Found  63129  primes
Finished in  102.9065528  seconds
\end{lstlisting}

Um nun den Effekt des GILs zu zeigen, wird die selbe Aufgabe durch mehrere Prozesse gel�st.
Hierzu bietet Python im \lstinline$multiprocessing$-Modul die Klasse \lstinline$Process$ an.
Die API, mit der neue Prozesse in Python erzeugt und gestartet werden, �hnelt der des
\lstinline$threading$-Moduls.
Demnach muss der Code nur minimal angepasst werden, um Primzahlen von Prozessen z�hlen zu lassen:
\label{prozesse:lst:primzahlen_prozesse}
\lstinputlisting[language=Python,linerange={1-2,25-43}]{chapters/nebenlaufigkeit/src/primzahlen_prozesse.py}

Wird dieses Programm ebenfalls ausgef�hrt, ist es m�glich, die Primzahlen schneller zu z�hlen.
Eine m�gliche Ausgabe kann wie folgt aussehen:
\label{prozesse:lst:prozess_ausgabe}
\begin{lstlisting}
Found  78498  primes
Found  70435  primes
Found  67883  primes
Found  65367  primes
Found  66330  primes
Found  64336  primes
Found  63129  primes
Found  63799  primes
Found  62712  primes
Found  62090  primes
Finished in  29.6911255  seconds
\end{lstlisting}

Beide Varianten erhalten das selbe Ergebniss, allerdings ist die Prozess-Variante kanpp 70
Sekunden schneller.
Ein gr��erer unterschied der beiden Varianten ist die Zeile \lstinline$if __name__ == ''__main__'':$.
Diese Zeile ist im\hyperref[prozesse:lst:primzahen_prozesse]{Prozess-Beispiel} notwendig, da der
Programmcode innerhalb des \lstinline$if$-Statements ansonsten jedesmal ausgef�hrt wird, wenn das
Modul importiert wird.
Wird ein neuer Prozess gestartet, l�dt der neue Python-Interpreter das Modul ein zweites mal.
Sobald er nun die Codezeile erreicht, in der der neue Prozess gestartet wurde, wird ein dritter Prozesse
gestartet, dessen Python-Interpreter das Modul ein drittes mal l�dt.
Somit werden solange Prozesse erzeugt, bis das System keine Ressourcen mehr zur Verf�gung hat.
Durch angabe des \lstinline$if$-Statements ist es garantiert, dass der enthaltene Code nur einmal
ausgef�hrt wird, wenn das Programm gestartet wird.
