% !TeX root = ../../pythonTutorial.tex
\subsection{Prozesse}
\label{prozesse:subsection:prozesse}

Die bisher betrachteten Beispiele und Aufgabe waren nicht CPU-gebunden, weshalb die Auswirkungen des
in Abschnit \ref{parallelit�t_in_python:subsection:parallelit�t_in_python} angesprochenen GILs nicht
ersichtlich wurden. 
Wird nun aber das folgende Beispiel betrachtet, in dem die Anzahl der Primzahlen im Zahlenbereich 1
bis 10 Millionen gesucht wird, ist dies nicht mehr der Fall:

\label{prozesse:lst:primzahlen_threads}
\lstinputlisting[language=Python]{chapters/nebenlaufigkeit/src/primzahlen_threads.py}

Jeder Thread bekommt hierbei einen Bereich aus 1 Milliionen Zahlen.
Es wird auch die Zeit gemessen, die die Threads ben�tigen, um die Primzahlen zu z�hlen.
Wird diese Programm ausgef�hrt, kann eine m�gliche Ausgabe wie folgt aussehen:
\label{prozesse:lst:thread_ausgabe}
\begin{lstlisting}
Found  67883  primes
Found  78498  primes
Found  70435  primes
Found  63799  primes
Found  65367  primes
Found  64336  primes
Found  66330  primes
Found  62712  primes
Found  62090  primes
Found  63129  primes
Finished in  102.9065528  seconds
\end{lstlisting}

Um nun den Effekt des GILs zu zeigen, wird die selbe Aufgabe durch mehrere Prozesse gel�st.
Hierzu bietet Python im \lstinline$multiprocessing$-Modul die Klasse \lstinline$Process$ an.
Die API, mit der neue Prozesse in Python erzeugt und gestartet werden, �hnelt der des
\lstinline$threading$-Moduls.
Demnach muss der Code nur minimal angepasst werden, um Primzahlen von Prozessen z�hlen zu lassen:
\label{prozesse:lst:primzahlen_prozesse}
\lstinputlisting[language=Python,linerange={1-2,25-43}]{chapters/nebenlaufigkeit/src/primzahlen_prozesse.py}

Wird dieses Programm ebenfalls ausgef�hrt, ist es m�glich, die Primzahlen schneller zu z�hlen.
Eine m�gliche Ausgabe kann wie folgt aussehen:
\label{prozesse:lst:prozess_ausgabe}
\begin{lstlisting}
Found  78498  primes
Found  70435  primes
Found  67883  primes
Found  65367  primes
Found  66330  primes
Found  64336  primes
Found  63129  primes
Found  63799  primes
Found  62712  primes
Found  62090  primes
Finished in  29.6911255  seconds
\end{lstlisting}

Beide Varianten erhalten das selbe Ergebniss, allerdings sind die Prozesse kanpp 70 Sekunden schneller.
Ein gr��erer unterschied der beiden Varianten ist die Zeile \lstinline$if __name__ == ''__main__''$.
Diese Zeile ist im \hyperref[prozesse:lst:primzahen_prozesse]{Prozess-Beispiel} notwendig, da der
Programmcode innerhalb des \lstinline$if$-Statements ansonsten jedesmal ausgef�hrt wird, wenn das
Modul importiert wird.
Wird ein neuer Prozess gestartet, l�dt der neue Python-Interpreter das Modul ein zweites mal.
Sobald er nun die Codezeile erreicht, in der der neue Prozess gestartet wurde, wird ein dritter Prozesse
gestartet, dessen Python-Interpreter das Modul ein drittes mal l�dt.
Somit werden solange Prozesse erzeugt, bis das System keine Ressourcen mehr zur Verf�gung hat.
Durch angabe des \lstinline$if$-Statements ist es garantiert, dass der enthaltene Code nur einmal
ausgef�hrt wird, wenn das Programm gestartet wird.

\subsubsection{Prozess Objekte}
\label{prozesse:subsubsection:prozess_objekte}

Wie aus dem betrachteten \hyperref[prozesse:lst:primzahen_prozesse]{Prozess-Beispiel} erkenntlich ist,
ist die Verwendung der \lstinline$Process$-Klasse sehr stark an die der \lstinline$Thread$-Klasse angelehnt.
Genau genommen gibt es jede Methode von \lstinline$Thread$ auch in \lstinline$Process$.
Sogar die Konstruktoren sind identisch.
Der \lstinline$group$-Parameter des \lstinline$Process$-Konstruktors existert allerdings nur zur
Kompatibilit�t zum Konstruktor der \lstinline$Thread$-Klasse.
Der Aufruf von \lstinline$join()$ gibt immer \lstinline$Non$ zur�ck, auch falls der optionale Timeout
abgelaufen ist.
Um zu pr�fen, ob der entsprechende Prozess tats�chlich beendet wurde, ist �ber seinen
\lstinline$exitcode$
einsehbar.
Es ist zu bemerken, dass ein D�mon-Prozess keine weiteren Kindprozesse starten kann.
Sobald sich sein Elternprozess beendet, wird er terminiert.
Zus�ztlich bietet die \lstinline$Process$-Klasse noch weitere Attribute und Methoden an.
�ber das \lstinline$pid$-Attribut kann die ID des jeweiligen Prozesses abgefragt werden. 
Durch \lstinline$exitcode$ kann der entsprechende abgefragt werden, mit welchem Status sich der Prozess
beendet hat.
Wurde er noch nicht beendet, hat \lstinline$exitcode$ den Wert \lstinline$None$.
Tr�gt \lstinline$exitcode$ einen negativen Wert, so bedeutet das, dass der Prozess durch ein Signal
beendet wurde.
Ein Wert von \lstinline$-N$ entspricht dann dem Signal \lstinline$N$.
Beim Attribut \lstinline$authkey$ handelt es sich um einen Byte-String, welcher f�r gewisse
Authentifizierungen genutzt wird und standardm��ig den Wert des Elternprozesses �bernimmt.
Genauere Informationen zur Verwendung von \lstinline$authkey$ sind in \cite{pythondokuprozesse}
 zu finden.
Die beiden Methoden \lstinline$terminate()$ und \lstinline$kill()$ beenden den jeweiligen Prozess, nicht
aber dessen Kindprozesse.
Verwendet der Prozess \lstinline$Locks$, \lstinline$Semaphoren$, \lstinline$Queues$ oder \lstinline$Pipes$,
so kann ein Aufruf der beiden Methoden dazu f�hren, dass die verwendeten Objekte unnutzbar werden 
und andere Prozesse in einen Deadlock geraten.
Unter Windows wird zum Beenden der Prozesse \lstinline$TerminateProcess()$ aufgerufen.
Unter Unix-Systemen sendet \lstinline$terminate()$ das \lstinline$SIGTERM$ Signal, w�hrend
\lstinline$kill()$ das Signal \lstinline$SIGKKILL$ sendet.
Durch Aufruf der \lstinline$close()$-Methode werden alle Resourcen des jeweiligen Prozesses freigegeben,
falls er bereits beendet ist.
Andernfalls wird ein \lstinline$ValueError$ geworfen.
Nachdem \lstinline$close()$ erfolgreich zur�ckgekehrt ist, wird beim Zugriff auf die meisten Methoden 
und Attributen von \lstinline$Process$ ebenfalls ein \lstinline$ValueError$ geworfen.

\warning{
Die Methoden \lstinline$start()$, \lstinline$join()$, \lstinline$is_alive()$, \lstinline$terminate()$ und
\lstinline$exitcode$ sollten immer nur vom erzeugenden Prozess aufgerufen werden.
}

Das \lstinline$multiprocessing$-Modul unterst�tzt, je nach Betriebssystem, drei verschiedene Arten einen
Prozess zu starten.
Bei diesen drei Arten handelt es sich um \lstinline$spawn$, \lstinline$fork$ und \lstinline$forkserver$,
welche sich folgenderma�en unterscheiden:

\begin{enumerate}
\item \lstinline$spawn$: \newline
Diese Art der Prozesserzeugung starten einen neuen Python Interpreter. 
Es werden nur diejenigen Resourcen an den neuen Prozess vererbt, die zum Ausf�hren seiner 
\lstinline$run()$-Methode n�tig sind.
Im Vergleich zu den beiden anderen Varianten ist diese eher langsam.
Sie ist unter Unix und Windows Systemen verf�gbar und der Standard unter Windows.

\item \lstinline$fork$: \newline
Durch diese Startmethode wird ein Fork des aktuellen Python Interpreters durch den Aufruf von 
\lstinline$os.fork()$ erstellt.
Das hei�t, dass der erzeugte Kindprozess effektiv identisch zum Elternprozess ist. 
Alle Resourcen werden vom Elternprozess geerbt.
Werden erzeugen Multithreaded-Prozesse mit dieser Startmethode Kindprozesse, kann es sehr schnell
problematisch werden.
Unter Windows ist diese Startmethode nicht verf�gbar, unter Unix Systemen ist sie die Standardvariante.

\item \lstinline$forkserver$: \newline
Wurde beim Starten des Programms diese Variante zum Starten von Prozessen gew�hlt, wird ein Server 
gestartet.
Immer wenn ein neuer Prozess erzeugt werden soll, verbindet sich der erzeugende Prozess mit diesem
Server und fordert an, einen Fork erstellen zu lassen.
Hierbei werden nur die notwendigen Resourcen an den Kindprozess vererbt.
Da es sich bei dem Fork-Server um einen Single-Threaded-Prozess handelt, ist ein Aufruf von 
\lstinline$os.fork()$ unproblematisch.
Diese Variante wird nur von Unix Systemen unterst�tzt, die auch das �bergeben von Dateideskriptoren 
�ber Unix-Pipes unterst�tzen.
\end{enumerate}

Um eine der drei Startmethoden zu w�hlen kann die \lstinline$set_start_methode()$-Methode aufgerufen 
werden.
Sie sollte nur innerhalb von \lstinline$if __name__ ==$ \lstinline$''__main__''$ aufgerufen werden und nie mehrmals
in einem Programm.
Die \lstinline$spwan$ Startmethode wird also wie im
\hyperref[prozesse:lst:prozess_start_methode]{folgenden Beispiel} gezeigt ausgew�hlt.

\label{prozesse:lst:prozess_start_methode}
\lstinputlisting[language=Python,linerange={1-2,11-16}]{chapters/nebenlaufigkeit/src/prozess_start_methode.py}

Sollen mehrere Prozesse mit unterschiedlichen Startmethoden gestartet werden, so kann alternativ
\lstinline$get_context()$ aufgrufen werden, um ein \lstinline$Context$-Objekt zu erhalten.
Hierbei ist darauf zu achten, dass Objekte, welche mit einem \lstinline$Context$ erzeugt wurden, nicht
immer kompatibel mit Prozessen sind, welche mit einem anderen \lstinline$Context$ gestartet wurden.
So ist ein \lstinline$Lock$-Objekt, welches mit einem \lstinline$fork Context$ erzeugt wurde, nicht mit
Prozessen kompatibel, die mittels der \lstinline$spawn$ oder der \lstinline$forkserver$ Startmethode 
erzeugt wurde.
Wie ein Prozess mit einer bestimmten Startmethode durch ein \lstinline$Context$-Objekt erzeugt wird,
ist im folgenden \hyperref[prozesse:lst:prozess_start_methode_context]{Beispiel} gezeigt.

\label{prozesse:lst:prozess_start_methode_context}
\lstinputlisting[language=Python,linerange={1-2,17-22}]{chapters/nebenlaufigkeit/src/prozess_start_methode.py}
