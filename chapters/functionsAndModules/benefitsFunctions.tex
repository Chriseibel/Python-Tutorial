% !TeX root = ../../pythonTutorial.tex
\section{Vorteile von Funktionen}
\label{benefitsFunctions}
Warum benutzen Programmierer Funktionen? Diese bieten eine Vielzahl an Vorteilen, wie z.B.

\begin{itemize}
	\item das Aufteilen von komplexen in mehrere simple Aufgaben
	\item das Verhindern von Code-Duplikationen
	\item bessere Lesbarkeit/Erweiterbarkeit/Ver�nderbarkeit
	\item vereinfachtes Debugging
\end{itemize}

\subsection{Aufteilen von komplexen Aufgaben}
\label{benefitsFunctions:subsection:splitComplexTask}
Bei komplexen Aufgaben kommt es schnell zu einer hohen Anzahl an
Code-Zeilen. Das kann f�r den Programmierer und die Kollegen, die an den Aufgaben mitwirken, problematisch sein.
Der Code ist schlecht zu lesen und die Fehlersuche schwierig.
Schritte, die immer wieder gebraucht werden, w�ren das Abfangen, Auslesen, Interpretieren und Aufbereiten von Daten. Anf�nger k�nnten den Fehler machen, diese Aufgaben in einer gro�en komplexen Funktion zu schreiben.

\begin{lstlisting}[language=Python, label=benefitsFunctions:lst:badComplexFunction]
# Beispiel f�r schlechten Umgang mit komplexen Prozessen

def processData(source):
  ...
  return finalData
\end{lstlisting}

Es ist m�hsam herauszufinden, was genau in dieser Funktion passiert.
Ein Beispiel zum Aufteilen des Code-Blocks k�nnte weiterhelfen.

\begin{lstlisting}[language=Python, label=benefitsFunctions:lst:goodComplexFunction]
# Aufteilung des komplexen Prozesses
# in mehrere kleine Prozesse

def processData(source):
	rawData = readData(source)
	parsedData = parseData(rawData)
	editedData = editData(parsedData)
	finalData = sortData(editedData)
	return finalData
\end{lstlisting}

\subsection{Reduktion von Code-Duplikationen}
\label{benefitsFunctions:subsection:codeDuplication}
Bestimmte Prozesse werden beim Programmieren immer wieder ben�tigt.
Bei der Arbeit mit Datens�tzen ist es �blich, diese nach gewissen Kriterien zu sortieren. 
In einer Datenbank ist die Sortierung nach der Identifikationsnummer vorteilhaft.
Diese \emph{Funktion} soll nicht f�r jeden Datensatzaufruf dupliziert werden m�ssen.
Daher wird dieser Prozess in einer Funktionen gespeichert, auf die wir von verschiedenen Positionen im Programm zugreifen k�nnen.

\subsection{Bessere Lesbarkeit, Erweiterbarkeit, Ver�nderbarkeit}
\label{benefitsFunctions:subsection:benefits}
Wie in Unterabschnitt \ref{benefitsFunctions:subsection:splitComplexTask} zu sehen ist, bringt das Aufteilen des Codes in spezifische Funktionen eine bessere Lesbarkeit mit sich.
Der Nutzer muss den Code nicht erst interpretieren. Bei intelligent gew�hlten Funktionsnamen versteht er, was in der Funktion passiert. Besonders beim Debuggen kann das Vorteile mit sich bringen, da der Programmierer nicht an den falschen Stellen suchen muss.
\begin{lstlisting}[]
array = calculateArray()

sortedArray = quickSort(array)
\end{lstlisting}

In diesem Beispiel wei� der Nutzer, dass das Array durch einen QuickSort-Algorithmus sortiert wird.
Sollte nun auffallen, dass es sich um falsche Werte handelt, muss der Programmierer nur die calculateArray-Funktion ansehen, sind die Werte falsch sortiert, so wird die quickSort-Funktion
n�her betrachtet.

Durch das Kapseln von Prozessen in einzelne Funktionen sind diese auch einfach erweiterbar und ver�nderbar.
W�re der Code nur dupliziert worden, m�sste der Nutzer diesen an allen Stellen �ndern.
Da das Programm aber an diesen Stellen nur die Funktion aufruft, muss nur diese Funktion erweitert oder ver�ndert werden.

