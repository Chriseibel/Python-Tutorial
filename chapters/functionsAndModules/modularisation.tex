% !TeX root = ../../pythonTutorial.tex

\section{Modularisierung}
\label{modularisation}
Unter Modularisierung versteht man die Aufteilung des Programmcodes in Module.
Ein Modul stellt dabei verschiedene Datentypen, Funktionen und Werte bereit.
Durch das Auslagern dieser Elemente in Module wird das Projekt besser strukturiert.
Zudem wird die Lesbarkeit des Quellcodes durch das Auslagern von Funktionen verbessert.

Bei den Modulen wird zwischen lokalen und globalen Modulen unterschieden.
W�hrend lokale Module nur in einem Projekt genutzt werden, k�nnen globale Module projekt�bergreifend verwendet werden.
Ein Beispiel f�r ein globales Modul ist das Modul \emph{math} der Standardbibliothek, welches mathematische Funktionen und Konstanten bereitstellt.

Ob ein Modul lokal oder global ist, h�ngt vom Speicherort ab. Ein lokales Modul ist in der Regel im Programmverzeichnis oder in einem Unterverzeichnis hinterlegt.
Ein globales Modul wird in einem bestimmten Verzeichnis der Python-Installation angelegt.
Hierzu geh�rt beispielsweise das Verzeichnis \emph{site-packages}, in dem auch einige Module von Drittanbietern installiert werden.

\subsection{Erstellung eines lokalen Moduls}
\label{modularisation:subsection:createLocalModule}
Zur Erstellung eines lokalen Moduls wird eine Datei \emph{myModul.py} im Programmverzeichnis angelegt:
\begin{lstlisting}[language=Python]
def hello(constantName):
	print("Hello, " + constantName)
    
MAX = "Maximilian"
\end{lstlisting}
Der Inhalt dieser Datei  f�hrt keinerlei Code aus, sondern stellt lediglich eine Funktion und eine Konstante bereit, welche sp�ter im Hauptprogramm genutzt werden k�nnen.

\subsection{Module verwenden}
\label{modularisation:subsection:useModule}
Um lokale und globale Module verwenden zu k�nnen, m�ssen diese zun�chst im Hauptprogramm eingebunden werden.
Hierzu wird die \emph{import}-Anweisung verwendet. Prinzipiell ist es egal, wo dies im Quellcode geschieht.
Es empfiehlt sich jedoch alle Module zu Beginn des Quelltextes einzubinden.
Die \emph{import}-Anweisung besteht aus dem Schl�sselwort \emph{import}, gefolgt vom Namen des Moduls.

\begin{lstlisting}[language=Python]
# Einbinden des zuvor erstellten Moduls

import myModule
\end{lstlisting}

Zu Beachten ist, dass beim Einbinden von Modulen die Dateiendung entf�llt.
Mit einer \emph{import}-Anweisung k�nnen auch mehrere Module eingebunden werden.
Hierzu werden die Modulnamen hinter dem Schl�sselwort - durch Kommas getrennt - aufgelistet:
\begin{lstlisting}[language=Python]
import myModule, math

# anstelle von:

import myModule
import math
\end{lstlisting}

Durch das Einbinden eines Moduls wird ein neuer Namensraum mit dem Modulnamen erzeugt.
�ber den Namensraum k�nnen alle Elemente des Moduls angesprochen werden:
\begin{lstlisting}[language=Python]
import myModule

myModule.hello("Welt")
myModule.hello(myModul.MAX)
\end{lstlisting}


Ausgabe:
\begin{lstlisting}[language=Python]
Hello, World
Hello, Maximilian
\end{lstlisting}

Mit Hilfe der \emph{import/as}-Anweisung ist es m�glich den Namen des Namensraums zu ver�ndern:
\begin{lstlisting}[language=Python]
import myModule as myBib
myBib.hello(World)
\end{lstlisting}

\tip{
Es gilt zu beachten, dass nach dieser Anweisung das Modul \emph{myModule} ausschlie�lich �ber den Namensraum \emph{myBib} zu erreichen ist.}

Python bietet zudem die M�glichkeit \emph{einzelne} Elemente aus einem Modul zu importieren.
Hierzu wird die \emph{from/import}-Anweisung verwendet:
\begin{lstlisting}[language=Python]
from math import sin, cos, tan, sinh, cosh, tanh
\end{lstlisting}

Falls die Liste der zu importierenden Elemente etwas l�nger ausfallen sollte, kann die Aufz�hlung in mehreren Zeilen erfolgen, um so die Lesbarkeit des Codes zu verbessern.
Zur Realisierung m�ssen zun�chst die einzubindenden Elemente in runden Klammern gesetzt werden.
Anschlie�end k�nnen beliebig viele Zeilenumbr�che innerhalb der runden Klammern erfolgen.
Folgendes Beispiel zeigt eine \emph{from/import}-Anweisung mit einem Zeilenumbruch:
\begin{lstlisting}[language=Python]
from math import (sin, cos, tan,
                    sinh, cosh, tanh)
\end{lstlisting}

Mit einem Stern k�nnen \emph{alle} Elemente des Moduls importiert werden:
\begin{lstlisting}[language=Python]
from myModule import *
\end{lstlisting}

Es gilt zu beachten, dass bei der \emph{from/import}-Anweisung kein eigener Namensraum erzeugt wird.
Die Elemente werden stattdessen in den \emph{globalen} Namensraum eingebunden.
Dies hat zur Folge, dass der Namensraum bei einem Zugriff auf ein Element nicht mehr angegeben werden muss:
\begin{lstlisting}[language=Python]
from myModul import hello
hello("World")		# statt: myModule.hello("World")
\end{lstlisting}


Da die Elemente bei der \emph{from/import}-Anweisung in den globalen Namensraum eingebunden werden, k�nnen bereits vorhandene Referenzen kommentarlos �berschrieben werden.

\begin{lstlisting}[language=Python]
pi = 42
print("pi hat den Wert:")
print(pi)

from math import pi		# pi wird �berschrieben
print("pi hat den Wert:")
print(pi)
\end{lstlisting}

Ausgabe:
\begin{lstlisting}[language=Python]
pi hat den Wert:
42
pi hat den Wert:
3.141592653589793
\end{lstlisting}

Um derartige Fehler zu minimieren, sollte die \emph{from/import}-Anweisung sparsam verwendet werden - also nur dann, wenn einzelne Elemente aus einem Modul ben�tigt werden.
Wenn alle oder fast alle Elemente aus einem Modul ben�tigt werden, empfiehlt es sich, die \emph{import}-Anweisung zu verwenden.
