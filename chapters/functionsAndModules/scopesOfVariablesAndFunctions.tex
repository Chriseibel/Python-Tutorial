% !TeX root = ../../pythonTutorial.tex
\section{G�ltigkeitsbereich von Variablen und Funktionen}
\label{scopesOfVariablesAndFunctions}
Je nachdem, wie und wo Funktionen definiert sind, k�nnen diese zu anderen Ergebnissen f�hren. Zur Verdeutlichung folgt ein einfaches Beispiel (\ref{scopesOfVariablesAndFunctions:lst:simpleScope}).

\begin{lstlisting}[language=Python, label=scopesOfVariablesAndFunctions:lst:simpleScope]
# Beispiel zu G�ltigkeitsbereichen

def myFunction():
    # lokaler G�ltigkeitsbereich der Funktion
    a = 1
    print('myFunction:', a)

# globaler G�ltigkeitsbereich
a = 0
myFunction()
print('global:', a)

# Ausgabe
# myFunction: 1
# global: 0
\end{lstlisting}

In beiden Bereichen benutzen wir die Variable \lstinline$a$.
Die Funktion wird nach dem Initialisieren der Variable aufgerufen.
Warum erhalten wir also zwei unterschiedliche Werte?

Der Grund: es handelt sich nicht um die gleiche Variable, da beide Variablen in verschiedenen G�ltigkeitsbereichen definiert werden. Lie�en wir die lokale Zuweisung aus, w�rde zweimal der Wert 0 ausgegeben werden.
Python sucht nach dem n�chstm�glichen G�ltigkeitsbereich:
\randnotiz{G�ltigkeits-
	bereiche}
\lstinline$lokal$, \lstinline$umschlie�end$, \lstinline$global$ und \lstinline$built-in$.

Nun ein Beispiel mit verschachtelten Funktionen:

\begin{lstlisting}[language=Python, label=scopesOfVariablesAndFunctions:lst:enclosingScope]
# Verschachtelte Funktionen

def enclosing():
    a = 1

    def innerFunction():
        a = 2
        print('innerste:', a)

    innerFunction()
    print('umschlie�end:', a)

a = 0
enclosing()
print('global:', a)

# Ausgabe
# innerste: 2
# umschlie�end: 1
# global: 0
\end{lstlisting}

\kontrollfrage{
	\item[\kontroll] Welche G�ltigkeitsbereiche gibt es in Python?
	\item[\kontroll] Bei der Nutzung einer Variable sucht Python nach dem n�chstm�glichen G�ltigkeitsbereich. Wie ist die Reihenfolge der G�ltigkeitsbereiche?
}

\subsection{Statements zu G�ltigkeitsbereichen - \mbox{global und nonlocal}}
\label{scopesOfVariablesAndFunctions:subsection:statements}
Nicht nur durch die Positionen werden G�ltigkeitsbereiche definiert, auch durch die Schl�sselw�rter \lstinline$global$ und \lstinline$nonlocal$ k�nnen wir den G�ltigkeitsbereich bestimmen.

Durch \randnotiz{nonlocal-Statement} \lstinline$nonlocal$ wird eine Variable auf die n�chst umschlie�ende Definition festgelegt (\ref{scopesOfVariablesAndFunctions:lst:nonlocalStatement}). 

\begin{lstlisting}[language=Python, label=scopesOfVariablesAndFunctions:lst:nonlocalStatement]
# Nonlocal Statement

def enclosing():
    a = 1

    def innerFunction():
        nonlocal a
        a = 2
        print('innerste:', a)

    innerFunction()
    print('umschlie�end:', a)

a = 0
enclosing()
print('global:', a)

# Ausgabe:
# innerste: 2
# umschlie�end: 2
# global: 0
\end{lstlisting}

\warning{W�rde in der enclosing-Funktion \lstinline$a$ auf \lstinline$nonlocal$ gesetzt, dann k�me es zu einer \lstinline$Exception$, da die n�chste Ebene global ist.
}

Das Gleiche \randnotiz{global-Statement} k�nnen wir mit dem globalen G�ltigkeitsbereich machen, wie in \ref{scopesOfVariablesAndFunctions:lst:globalStatement} gezeigt.

\begin{lstlisting}[language=Python, label=scopesOfVariablesAndFunctions:lst:globalStatement]
# Global Statement

def enclosing():
    a = 1

    def inner():
        global a
        a = 2
        print('innerste:', a)

    innereFunction()
    print('umschlie�end:', a)

a = 0
enclosing()
print('global:', a)

# Ausgabe
# innerste: 2
# umschlie�end: 1
# global: 2
\end{lstlisting}


W�hrend \lstinline$nonlocal$ nur den n�chst umschlie�enden G�ltigkeitsbereich w�hlt -  in welcher die Variable deklariert wurde - greift \lstinline$global$ immer auf den globalen G�ltigkeitsbereich zu.

\kontrollfrage{
	\item[\kontroll] Von welcher Konvention befreit uns die Nutzung von Statements bez�glich des G�ltigkeitsbereiches?
	\item[\kontroll] Was passiert, wenn eine Variable mit dem Statement \lstinline{nonlocal} gekennzeichnet wird, es sich beim n�chsten G�ltigkeitsbereich jedoch um den globalen handelt?
}
