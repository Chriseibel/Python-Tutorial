% !TeX root = ../pythonTutorial.tex
\section{Nebenl�ufigkeit}
\label{nebenl�ufigkeit:section:nebenl�ufigkeit}

Mit Nebenl�ufigkeit ist eine Eigenschaft von zwei oder mehr Aktivit�ten gemeint.
Eine beliebige Anzahl an Aktivit�ten wird als nebenl�ufig bezeichnet, wenn die
Reihenfolge der Ausf�hrung der einzelnen Aktivit�ten irrelevant f�r das Ergebnis ist.
Mit anderen Worten ist es egal, ob zu erst die eine Aktivit�t und dann die andere
Aktivit�t ausgef�hrt wird, oder umgekehrt, oder sogar beide zeitgleich.
Hierbei ist auf den Unterschied von zwischen Nebenl�ufigkeit und Parallelit�t zu achten.
W�hrend Nebenl�ufigkeit eine Eigenschaft darstellt, ist Parallelit�t eine m�gliche
Herangehensweise an das Ausf�hren von nebenl�ufigen Aktivit�ten.

Im folgenden Kapitel wird beschrieben, wie in Python durch Threads und Prozesse
nebenl�ufige Programmierung realisiert werden kann.
Python unterscheidet sich in Bezug auf Parallelit�t stark von anderen Programmiersprachen.
In Abschnitt \ref{parallelit�t_in_python:subsection:parallelit�t_in_python} wird auf diesen
Unterschied eingegangen.
Anschlie�end wird in Abschnitt \ref{threads:subsection:threads} beschrieben, wie in Python Nebenl�ufigkeit
durch Threads realisiert werden kann.
Neben der Erzeugung von Threads werden Synchronisations- und Steuerungsmechanismen besprochen.
Nachdem nun Nebenl�ufigkeit durch Threads erreicht wurde, wird sich in Abschnitt
\ref{prozesse:subsection:prozesse}
auf Prozesse bezogen.

% !TeX root = ../../pythonTutorial.tex
\subsection{Parallelit�t in Python}
\label{parallelit�t_in_python:subsection:parallelit�t_in_python}

In der Referenzimplementierung CPython des Python-Interpreters existiert ein Konstrukt,
welches eine echte parallele Ausf�hrung von Python-Code verhindert.
Bei diesem Konstrukt handelt es sich um das sogenannte Global-Interpreter-Lock, oder kurz GIL.
F�r die Verwendung eine GILs in Python sprechen mehrere Punkte.
Python wurde so entworfen, dass es leicht zu verwenden ist, um den
Entwicklungsprozess zu beschleunigen.
Ein GIL verhindert, dass sich mehrere Threads gleichzeitig in der Ausf�hrung befinden k�nnen,
was die Entwicklung von Multithreaded-Programmen erheblich erleichtert.
Weiterhin wurde der Funktionsumfang von Python durch viele in C geschriebene Erweiterungen erg�nzt.
Um Inkonsistenzen zu verhindern, ben�tigen diese C-Erweiterungen eine threadsichere
Speicherverwaltung, welche durch das GIL garantiert ist.
Die Verwendung des GILs erleichtert auch die Integration von nicht threadsicheren C-Bibliotheken.
Da das Einbinden von C-Bibliotheken durch das GIL leicht zu realisieren ist, existieren so
viele Erweiterungen zu Python, welche zur weiten Verbreitung von Python f�hrten.

Durch die Verwendung des GILs ist eine parallele Programmierung in Python
allerdings nicht g�nzlich ausgeschlossen.
Lediglich CPU-gebundene Aufgaben sind hierdurch betroffen.
I/O-gebundene Aufgaben, wie zum Beispiel das Anfragen von Daten aus einer Datenbank, oder
das Abfragen von Benutzereingaben, k�nnen auch trotz des GILs echt parallel ausgef�hrt werden.
Durch das Verwenden von mehreren Prozessen ist es auch m�glich parallele
Ausf�hrung von Python-Code zu erreichen.
Dies funktioniert, da jeder Python-Prozess seinen eignen Python-Interpreter und somit
auch ein eigenes GIL besitzt.

% !TeX root = ../../pythonTutorial.tex
\subsection{Threads}
\label{threads:subsection:threads}

In Python werden zwei APIs zur Verwendung von Threads angeboten, die Low-Level
API aus dem \_thread-Modul und die Higher-Level API aud dem threading-Modul.
Es wird sich an dieser Stelle auf das threading-Modul beschr�nkt, da es Intern auf
dem \_thread-Modul basiert und eine Schnittstelle anbietet, welche das Programmieren
von Multithreaded-Programmen erleichert.
Diese Schnittstelle ist an der Thread-Schnittstelle von Java angelehnt und sollte daher f�r
Java-Entwickler leicht zu verwenden sein.
Allerdings gibt es einige Unterschiede zwischen dem Python-Modul und der Java-Implementierung.
So sind Bedingungsvariablen und Locks seperate Objekte in Python und es ist auch
nur eine Teilmenge des Verhaltens eines Java-Threads in Python verf�gbar.
Ein Python-Thread kennt keine Priorit�ten und Thread-Gruppen und er kann nicht zerst�rt,
gestopped, angehalten, fortgesetzt oder unterbrochen werden.
Soweit vorhanden sind die statischen Methoden aus der Java-Thread-Klasse auf Modul-Ebene
in Python implementiert.

\subsubsection{Thread Objekte}
\label{threads:subsubsection:thread_objekte}

Es gibt zwei M�glichkeiten einen Thread zu erzeugen. Entweder wird dem Konstruktor ein
aufrufbares Objekt �bergeben,

\label{threads:lst:thread_erzeugung_callable}
\lstinputlisting[language=Python,firstline=4,lastline=9]{chapters/nebenlaufigkeit/src/beispiel_thread_erzeugung.py}

oder die \lstinline$run()$-Methode wird in einer von \lstinline$Thread$ abgeleiteten Klasse �berschrieben.

\label{threads:lst:thread_erzeugung_subclass}
\lstinputlisting[language=Python,firstline=12,lastline=18]{chapters/nebenlaufigkeit/src/beispiel_thread_erzeugung.py}

Der Konstruktor  der \lstinline$Thread$-Klasse bietet noch weitere Parameter an:
\begin{itemize}
    \item \lstinline$group$ sollte immer \lstinline$None$ sein.
    Es ist aktuell reserviert f�r sp�tere Erweiterungen.
    \item \lstinline$name$ setzt den Namen des Threads.
    \item \lstinline$args$ ist ein Tupel aus Parametern f�r das mit \lstinline$target$ definierte
    aufrufbare Objekt.
    \item \lstinline$kwargs$ ist ein Dictionary aus Schl�sselwort-Parametern f�r \lstinline$target$.
    \item \lstinline$deamon$ setzt die D�mon-Eigenschaft des Threads.
\end{itemize}

Es ist anzumerken, dass ein \lstinline$Thread$-Objekt bei seiner Erzeugung noch nicht gestartet wird.
Hierzu muss explizit die \lstinline$start()$-Methode aufgerufen werden.
Wurde ein Thread gestartet wird er als ''lebendig'' angesehen. 
Dies bleibt er solange, bis seine \lstinline$run()$-Methode verlassen wurde.
Hierbei macht es keinen Unterschied, ob sie regul�r verlassen wurde oder Aufgrund einer Exception.
Der aktuelle Status eines Threads kann mittels der \lstinline$is_alive()$-Methode abgefragt werden.
Soll auf das Beenden eines anderen Threads gewartet werden, so kann seine
\lstinline$join()$-Methode aufgerufen werden. 
Hiermit wird der aufrufende Thread blockiert, bis der andere beendet ist.
Die \lstinline$join$-Methode nimmt einen optionalen Parameter des Typen \lstinline$float$ entgegen,
der als Timeout in Sekunden dient.
Wird ein Timeout angegeben, ist es wichtig, dass nach dem \lstinline$join()$-Aufruf die Methode 
\lstinline$is_alive()$ aufgerufen wird.
Da \lstinline$join()$ immer \lstinline$None$ zur�ckgibt, ist es ansonsten nicht m�glich, zu wissen, ob
der Thread tats�chlich beendet wurde, oder nur der Timeout abgelaufen ist.

\kontrollfrage{
\item[\kontroll] Wie kann auf das Ende der Ausf�hrung eines Threads gewartet werden?
}

Jeder Thread besitzt einen Namen, welcher initial �ber den Konstruktor oder direkt �ber das
\lstinline$name$-Attribut gesetzt werden kann. 
Threads k�nnen als D�mon gekennzeichnet werden.
Sobald nur noch D�mon-Threads aktiv sind, wird das Python-Programm beendet.
Die D�mon-Eigenschaft kann initial �ber den Konstruktor gesetzt werden.
Wird kein Wert �bergeben, �bernimmt der Thread standardm��ig den Werte des erzeugenden Threads.
�ber das \lstinline$deamon$-Attribut eines Threads kann die Eigenschaft abgefragt und gesetzt werden.
Hierbei ist es wichtig, dass die Eigenschaft immer vor dem Aufruf der \lstinline$start()$-Methode
gesetzt wird.
Wird sie nach dem Starten des Threads ge�ndert, so wird ein \lstinline$RuntimeError$ geworfen.

\warning{
	D�mon-Threads werden sofort beendet, wenn keine normalen Threads mehr aktiv sind.
	Das hei�t, dass ihre Ressourcen wie zum Beispiel ge�ffnete Dateien oder Datenbanktransaktionen
	gegebenenfalls nicht ordentlich freigegeben werden.
	Um dies zu verhindern, sollten die betroffenen Threads nicht die D�mon-Eigenschaft besitzen und
	es sollten geeignete Signalisierungsmechanismen eingesetzt werden (siehe \lstinline$Event$-Objekte).
	%ggf. entfernen falls Kapitel nicht in finaler Abgabe
}

\uebung
\aufgabe{nebenlaufigkeit01}
\aufgabe{nebenlaufigkeit02}


\subsubsection{Synchronisation}
\label{threads:subsubsection:synchronisation}

Die meisten Anwendungen, in denen mehrere Threads zum Einsatz kommen, erfordern einen
Mechanismus, der die Zugriffe der einzelnen Threads auf gewisse Daten synchronisiert.
Hierdurch wird unteranderem vermieden, dass auf invaliden Datens�tzen gearbeitet wird, oder ein
Datenupdate verloren geht.
Im Folgenden wird ein Beispiel betrachtet, bei dem es zu Fehlern Aufgrund von fehlender
Synchronisationsmechanismen kommt.
Es werden anschlie�end neue Konstrukte eingef�hrt, welche die Fehler beheben werden.

Betrachtet wird nun die folgende \hyperref[threads:lst:counter_example]{\lstinline$Counter$-Klasse}.
Sie besitzt das Attribut \lstinline$count$, welches durch Aufruf von \lstinline$increment()$
in Einerschritten erh�ht wird. 

\label{threads:lst:counter_example}
\lstinputlisting[language=Python,firstline=4,lastline=9]{chapters/nebenlaufigkeit/src/beispiel_synchronisation_fehler.py}

Es ist weiterhin die \hyperref[threads:lst:incrementer_thread]{\lstinline$IncrementerThread$-Klasse}
gegeben, welche bei der Initialisierung ein \lstinline$Counter$-Objekt erwartet.
Dieser Thread ruft eine Millionen mal die \lstinline$increment()$-Methode des \lstinline$Counters$
auf und beendet sich anschlie�end.

\label{threads:lst:incrementer_thread}
\lstinputlisting[language=Python,firstline=12,lastline=19]{chapters/nebenlaufigkeit/src/beispiel_synchronisation_fehler.py}

F�r dieses Beispiel werden nun 10 \lstinline$IncrementerThreads$ erzeugt und gestartet.
Anschlie�end wird auf ihre Terminierung gewartet und dann der Wert des \lstinline$count$-Attributs des
\lstinline$Counters$ ausgegeben (vgl. Listing \ref{threads:lst:example_no_locks}).

\label{threads:lst:example_no_locks}
\lstinputlisting[language=Python,firstline=22]{chapters/nebenlaufigkeit/src/beispiel_synchronisation_fehler.py}

\kontrollfrage{
\item[\kontroll] Welche Ausgabe w�rde erwartet werden, wenn 10 Threads den Counter eine Millionen
mal inkrementieren?
}

Dieser Programmcode w�rde vermuten lassen, dass bei jeder Ausf�hrung der Wert 10000000 ausgegeben
wird, da jeder der 10 Threads den \lstinline$Counter$ eine Millionen mal inkrementiert.
Erstaunlicherwei�e werden allerdings bei mehrmaliger Ausf�hrung unterschiedliche Werte ausgegeben.
Diese k�nnen zum Beispiel wie folgt aussehen:

\label{threads:lst:beispiel_ausgabe}
\begin{lstlisting}
# M�gliche Ausgaben:
5237496
3561559
4089438
4526494
\end{lstlisting}

Dieses Ph�nomen l�sst sich so erkl�ren, dass die Operation in \lstinline$increment()$ nicht atomar ist.
Genau genommen werden in ihr drei Operationen, eine lesende, eine addierende und eine schreibende,
ausgef�hrt.
Somit kann es vorkommen, dass zum Beispiel der erste Thread den aktuellen \lstinline$count$-Wert liest
und dann die aktive Ausf�hrung an einen anderen Thread abgeben muss.
Dieser zweite Thread liest nun den selben \lstinline$count$-Wert wie der erste Thread, inkrementiert ihn
und schreibt den neuen Wert zur�ck in das Attribut.
Nun wechselt die aktive Ausf�rhung zur�ck zum ersten Thread, welcher noch den alten \lstinline$count$-Wert gelesen hat.
Dieser alte Wert wird nun erneut inkrementiert und zur�ckgeschrieben. 
Somit wurde der \lstinline$Counter$ effektiv nicht zweimal, sondern nur einmal inkrementiert.
Um dieses Verhalten zu verhindern, muss sichergestellt werden, dass die drei einzelnen Operationen
atomar ausgef�hrt werden.
Das hei�t, dass sie entweder ganz oder garnicht ausgef�hrt werden.

Als unterste Synchronisationsebene bietet Python die Klasse \lstinline$Lock$ an.
\randnotiz{Locks}
Ein Objekt dieser Klasse befindet sich immer in einem von zwei Zust�nden, es ist entweder offen
oder geschlossen.
Nach der Initialisierung befindet es sich zuerst im ge�ffneten Zustand.
Ein \lstinline$Lock$-Objekt stellt die beiden Methoden \lstinline$acquire()$ und \lstinline$release()$ zur
Verf�gung.
Wird \lstinline$acquire()$ auf einem offenen \lstinline$Lock$ aufgerufen, so begibt sich das \lstinline$Lock$
in den geschlossenen Zustand und die Methode kehrt sofort zur�ck.
Sollte die \lstinline$aquire()$-Methode aufgerufen werden, wenn sich das \lstinline$Lock$ im
geschlossenen Zustand befindet, so blockiert sie solange, bis \lstinline$release()$ in einem anderen Thread
aufgerufen wird und somit den Zustand des \lstinline$Locks$ wieder zu ge�ffnet �ndert.
Die blockierte \lstinline$aquire()$-Methode schlie�t dann das \lstinline$Lock$ wieder und kehrt zur�ck.
Wird auf einem offenen \lstinline$Lock$ die \lstinline$release()$-Methode aufgerufen, so wird ein
\lstinline$RuntimeError$ geworfen.
Falls mehrere Threads durch \lstinline$aquire()$ blockiert werden, wird nur ein Thread fortgesetzt, sobald
\lstinline$release()$ aufgerufen wurde.
Welcher der blockierten Threads fortgesetzt wird ist hierbei nicht definiert.

Wurde \lstinline$aquire()$ aufgerufen, sollte garantiert sein, dass auch \lstinline$release()$
aufgerufen wird.
Wird eine \lstinline$Exception$ geworfen, kann dies allerdings nicht immer garantiert sein.
Aus diesem Grund wird empfohlen, den durch das \lstinline$Lock$ gesch�tzten Programmcode in
einen \lstinline$try$-Block zu schreiben und den Aufruf von \lstinline$release()$ in den
\lstinline$finally$-Block zu schreiben:

\label{threads:lst:try_aquire_lock}
\lstinputlisting[language=Python,firstline=5,lastline=9]{chapters/nebenlaufigkeit/src/beispiel_lock_aquire_release.py}

Diese Variante ist allerdings etwas lang und unsch�n.
Da die \lstinline$Lock$-Klasse das Context-Management-Protokoll unterst�tzt, kann das Gleiche mit dem
\lstinline$with$-Statement erreicht werden.
Hierbei werden \lstinline$aquire()$ und \lstinline$release()$ automatisch aufgerufen:

\label{threads:lst:with_aquire_lock}
\lstinputlisting[language=Python,firstline=11,lastline=12]{chapters/nebenlaufigkeit/src/beispiel_lock_aquire_release.py}

Wird \lstinline$aquire()$ ohne Parameter aufgerufen, blockiert sie solange, bis \lstinline$release()$ 
aufgerufen wird.
Ist dies nicht gew�nscht, so kann auch der optionale Parameter \lstinline$blocking=False$ angegeben
werden. 
In diesem Fall kehrt \lstinline$aquire()$ sofort zur�ck, egal in welchem Zustand sich das \lstinline$Lock$
befindet.
Es muss nun der R�ckgabewert von \lstinline$aquire()$ betrachtet werden, um zu erfahren, ob das 
\lstinline$Lock$ offen oder geschlossen ist.
Ist das \lstinline$Lock$ bereits geschlossen wird der Wert \lstinline$False$ zur�ckgegeben.
Andernfalls wird \lstinline$True$ zur�ckgegeben und das \lstinline$Lock$ �ndert seinen Zustand zu
geschlossen.
Weiterhin ist es m�glich, einen Timeout mittels des optionalen Parametes \lstinline$timeout$ zu 
spezifizieren.
Hierbei kann eine beliebige Zeit in Sekunden als \lstinline$float$ Wert angegeben werden.
In diesem Fall blockiert \lstinline$aquire()$ maximale die spezifizierte Zeit.
Wurde in dieser Zeit das \lstinline$Lock$ erlangt, so gibt \lstinline$aquire()$ \lstinline$True$ zur�ck,
andernfalls \lstinline$False$.
Die Angabe eines Timeouts ist nur erlaubt, wenn der Parameter \lstinline$blocking$ den Wert
\lstinline$True$ besitzt.

\uebung
\aufgabe{nebenlaufigkeit03}
\aufgabe{nebenlaufigkeit04}

Um das Problem aus Aufgabe \ref{nebenlaufigkeit04} zu l�sen bietet Python eine weitere M�glichkeit
zur Synchronisation an.
\randnotiz{Reentrant Locks}
Hierbei handelt es sich um die \lstinline$RLock$-Klasse.
Das \lstinline$R$ steht f�r \lstinline$reentrant$, was auf deutsch Wiedereintritt bedeutet.
Im Gegensatz zu Objekten der \lstinline$Lock$-Klasse, die nie einem Thread zugeordnet werden, werden
Objekte der \lstinline$RLock$-Klasse an den Thread gebunden, der zuerst die \lstinline$aquire()$-Methode
aufruft.
Neben den beiden Zust�nden, die die \lstinline$Lock$-Klasse besitzt, merkt sich die \lstinline$RLock$-Klasse
nun auch, wie oft die \lstinline$aquire()$-Methode aufgerufen wurde.
Beim ersten Aufruf von \lstinline$aquire()$ merkt sich das \lstinline$RLock$-Objekt, welcher Thread die
Methode aufgerufen hat und setzt einen internen Z�hler auf 1.
Bei jedem weiteren Aufruf von \lstinline$aquire()$ des selben Threads wird der Z�hler inkrementiert.
Wird \lstinline$release()$ aufgerufen, so wird der Z�hler wieder dekrementiert.
Das \lstinline$RLock$ ist erst dann wieder offen, wenn \lstinline$release()$ genau so oft aufgerufen
wurde, wie \lstinline$aquire()$, und der interne Z�hler wieder auf 0 steht.
Ruft ein zweiter Thread die \lstinline$aquire()$-Methode auf, w�hrend der erste Thread das
\lstinline$RLock$ besitzt, so muss er warten, bis der Z�hler wieder auf 0 steht.
Es ist nun also m�glich einen gewissen Codeabschnitt auch rekursiv vor konkurrierenden Zugriffen 
zu sch�tzen.
Wie auch schon beim \lstinline$Lock$, k�nnen der \lstinline$aquire()$-Methode des \lstinline$RLocks$ 
die beiden optionalen Paramter \lstinline$blocking$ und \lstinline$timeout$ �bergegeben werden.

\uebung
\aufgabe{nebenlaufigkeit05}

Im Folgenden wird das Beispiel mit dem \lstinline$Counter$ und dem \lstinline$Incrementer-$
\lstinline$Thread$ etwas angepasst.
Der \lstinline$IncrementerThread$ soll nun den \lstinline$Counter$ wieder nur um eins erh�hen.
Weiterhin wird dem \lstinline$IncrementerThread$ ein Wert �bergeben, mit welchem gesteuert wird,
wann der Thread den \lstinline$Counter$ erh�ht. Den 10 erstellten \lstinline$IncrementerThreads$ 
wird nun eine Zahl von 0 bis 9 �bergeben. Der \lstinline$Counter$ soll von den einzelnen Threads immer
nur dann erh�ht werden, wenn der aktuelle Wert des \lstinline$Counters$ auf die Ziffer endet, die dem
Thread bei der Erzeugung �bergeben wurde.
Demnach m�ssen die \lstinline$IncrementerThreads$ auf einen bestimmten geteilten Zustand warten,
bevor sie \lstinline$increment()$ aufrufen d�rfen.
\randnotiz{Condition-Variable}
F�r einen solchen Anwendungsfall stellt Python die \lstinline$Condition$-Klasse zur Verf�gung.
Der Mechanismus, der hierdurch implementiert wird, ist allgemein als Condition Variable
(deutch Bedingungsvariable) bekannt.
Objekte der \lstinline$Condition$-Klasse sind immer einem \lstinline$Lock$- oder einem
\lstinline$RLock$-Objekt zugeordnet.
Dieses kann dem Konstruktor eines \lstinline$Condition$-Objektes �bergeben werden.
Wird kein Lock-Objekt �bergeben, erzeugt der Konstruktor ein neues.
Das so erzeugte \lstinline$Condition$-Objekt kann nun �berall wie das Lock-Objekt
verwendet werden, das Lock-Objekt muss nicht weiterhin zus�tzlich verwaltet werden.
Es kann nun also das \lstinline$RLock$ aus der \lstinline$Counter$-Klasse gegen eine Condition Variable
ausgetauscht werden.
Die neue \hyperref[threads:lst:counter_condition_variable_example]{\lstinline$Counter$-Klasse} sieht
dann wie folgt aus:

\label{threads:lst:counter_condition_variable_example}
\lstinputlisting[language=Python,firstline=4,lastline=15]{chapters/nebenlaufigkeit/src/beispiel_condition_variable.py}

Eine Condition Variable kann also wie ein einfaches Lock verwendet werden.
Die \lstinline$aquire()$- und \lstinline$release()$-Methoden verhalten sich hierbei, wie die des hinterlegten
Lock-Objektes.
Dar�ber hinaus bietet die \lstinline$Condition$-Klasse noch weitere Methoden an.
Diese Methoden d�rfen nur aufgerufen werden, wenn zuvor \lstinline$aquire()$ aufgerufen wurde.
Wurde von einem Thread \lstinline$aquire()$ aufgerufen, aber der aktuelle geteilte Zustand
nicht den gew�nschten Bedingungen entspricht, so wird \lstinline$wait()$ aufgerufen.
Die \lstinline$wait()$-Methode gibt das Lock wieder frei und blockiert den Thread, bis er
aufgeweckt wird.
Ein Thread wird durch Aufruf der \lstinline$notify()$- oder der \lstinline$notifiy_all()$
-Methode aufgeweckt.
Diese Methoden sollten immer dann aufgerufen werden, wenn der geteilte Zustand von einem Thread 
ge�ndert wurde.
Sobald ein Thread aufgeweckt wurde, fordert \lstinline$wait()$ wieder das Schloss an und kehrt
dann zur�ck.
Nachdem \lstinline$wait()$ zur�ckgekehrt ist, sollten die Bedingungen an den geteilten Zustand auf 
jeden Fall wieder gepr�ft werden, da eine unbestimmte Zeit zwischen dem Aufruf von \lstinline$notify()$
oder \lstinline$notify_all()$ und dem Zur�ckkehren von \lstinline$wait()$ vergehen kann.
Weiterhin ist es m�glich \lstinline$wait()$ den optionalen Parameter \lstinline$timeout$ mitzugeben.
L�uft diese Zeit ab, bevor ein anderer Thread \lstinline$notify()$ oder \lstinline$notify_all()$ aufruft,
kehrt \lstinline$wait()$ mit dem R�ckgabewert \lstinline$False$ zur�ck.

\tip{
Um die Entscheidung zwischen \lstinline$notify()$ und \lstinline$notify_all()$ zu erleichtern, sollte die Frage
gestellt werden, ob die �nderung des geteilten Zustands f�r nur einen Thread oder mehrere Threads
interessant ist.
}

Durch einen Aufruf von \lstinline$notify_all()$ werden alle Threads, die \lstinline$wait()$ auf dem
entsprechenden \lstinline$Condition$-Objekt aufgerufen haben, aufgeweckt.
Mit \lstinline$notify()$ wird nur ein Thread aufgeweckt.
Es kann ein optionaler Parameter \lstinline$n$ an \lstinline$notify()$ �bergeben werden, der angibt, wie
viele Threads aufgeweckt werden sollen.

\warning{
	Durch den Aufruf von \lstinline$notify()$ und \lstinline$notify_all()$ wird das Lock nicht freigegeben.
	Das hei�t, dass Threads, die \lstinline$wait()$ aufgerufen haben, erst dann wieder aufwachen, wenn
	der Thread, der \lstinline$notify()$ oder \lstinline$notify_all()$ aufgerufen hat das Lock wieder 
	explizit frei gibt.
}

\kontrollfrage{
\item[\kontroll] Ist es f�r das \lstinline$Counter$ und \lstinline$IncrementerThread$ Beispiel ausreichend,
\lstinline$notify()$ aufzurufen?
}

Das \lstinline$Counter$ und \lstinline$IncrementerThread$ Beispiel w�rde nicht funktionieren, wenn 
immer nur ein beliebiger Thread geweckt wird. 
Wird der \lstinline$Counter$ inkrementiert, muss \lstinline$notify_all()$ aufgerufen werden,
um einen Deadlock zu vermeiden.
Dies liegt daran, das immer nur genau ein wartender Thread fortschreiten kann und es ist nicht
garantiert werden kann, dass genau dieser Thread aufgeweckt wird.

Wie das generische \hyperref[threads:lst:producer_consumer]{Producer-Consumer-Pattern}
mithilfe von Condition Variablen implementiert werden kann, ist im folgenden Listing
gezeigt (vgl. \cite{pythondokuthreads}).
Die Consumer warten so lange, bis der geteilte Zustand der Bedingung entspricht.
In diesem Fall hei�t das, dass mindestens ein Element verf�gbar ist.
Sobald ein Producer ein Element erstellt hat, ruft er \lstinline$notify()$ auf.
Somit wird genau ein Consumer aufgeweckt.
In diesem Beispiel reicht es vollkommen aus, nur einen Consumer aufzuwecken.

\label{threads:lst:producer_consumer}
\lstinputlisting[language=Python,firstline=21,lastline=30]{chapters/nebenlaufigkeit/src/beispiel_producer_consumer.py}

Der Consumer ruft \lstinline$wait()$ innerhalb der \lstinline$while$-Schleife auf und pr�ft
jedes mal seine Bedingung.
Dies ist notwendig, da sich der Zustand zwischen dem Aufruf von \lstinline$notify()$ und dem 
Zur�ckkehren von \lstinline$wait()$ erneut �ndern kann. 
Diese Problematik ist inh�rent in der Multithreaded-Programmierung.
Die \lstinline$Condition$-Klasse bietet neben \lstinline$wait()$ eine weitere Methode an, die das Testen
der Bedingung automatisieren kann.
Bei dieser Methode handelt es sich um \lstinline$wait_for()$.
Ihr Paramater \lstinline$predicate$ nimmt ein Callable-Objekt entgegen, welches einen boolischen Wert
zur�ck gibt.
Es kann zudem ein Timeout angegeben werden, der sich wie bei \lstinline$wait()$ verh�lt.
Wird \lstinline$wait_for()$ verwendet, �ndert sich das
\hyperref[threads:lst:producer_consumer_wait_for]{Producer-Consumer-Pattern Beispiel}
folgender Ma�en (vgl. \cite{pythondokuthreads}):

\label{threads:lst:producer_consumer_wait_for}
\lstinputlisting[language=Python,firstline=32,lastline=35]{chapters/nebenlaufigkeit/src/beispiel_producer_consumer.py}

\uebung
\aufgabe{nebenlaufigkeit06}


% !TeX root = ../../pythonTutorial.tex
\subsection{Prozesse}
\label{prozesse:subsection:prozesse}

Die bisher betrachteten Beispiele und Aufgabe waren nicht CPU-gebunden, weshalb die Auswirkungen des
in Abschnit \ref{parallelit�t_in_python:subsection:parallelit�t_in_python} angesprochenen GILs nicht
ersichtlich wurden. 
Wird nun aber das folgende Beispiel betrachtet, in dem die Anzahl der Primzahlen im Zahlenbereich 1
bis 10 Millionen gesucht wird, ist dies nicht mehr der Fall:

\label{prozesse:lst:primzahlen_threads}
\lstinputlisting[language=Python]{chapters/nebenlaufigkeit/src/primzahlen_threads.py}

Jeder Thread bekommt hierbei einen Bereich aus 1 Milliionen Zahlen.
Es wird auch die Zeit gemessen, die die Threads ben�tigen, um die Primzahlen zu z�hlen.
Wird diese Programm ausgef�hrt, kann eine m�gliche Ausgabe wie folgt aussehen:
\label{prozesse:lst:thread_ausgabe}
\begin{lstlisting}
Found  67883  primes
Found  78498  primes
Found  70435  primes
Found  63799  primes
Found  65367  primes
Found  64336  primes
Found  66330  primes
Found  62712  primes
Found  62090  primes
Found  63129  primes
Finished in  102.9065528  seconds
\end{lstlisting}

Um nun den Effekt des GILs zu zeigen, wird die selbe Aufgabe durch mehrere Prozesse gel�st.
Hierzu bietet Python im \lstinline$multiprocessing$-Modul die Klasse \lstinline$Process$ an.
Die API, mit der neue Prozesse in Python erzeugt und gestartet werden, �hnelt der des
\lstinline$threading$-Moduls.
Demnach muss der Code nur minimal angepasst werden, um Primzahlen von Prozessen z�hlen zu lassen:
\label{prozesse:lst:primzahlen_prozesse}
\lstinputlisting[language=Python,linerange={1-2,25-43}]{chapters/nebenlaufigkeit/src/primzahlen_prozesse.py}

Wird dieses Programm ebenfalls ausgef�hrt, ist es m�glich, die Primzahlen schneller zu z�hlen.
Eine m�gliche Ausgabe kann wie folgt aussehen:
\label{prozesse:lst:prozess_ausgabe}
\begin{lstlisting}
Found  78498  primes
Found  70435  primes
Found  67883  primes
Found  65367  primes
Found  66330  primes
Found  64336  primes
Found  63129  primes
Found  63799  primes
Found  62712  primes
Found  62090  primes
Finished in  29.6911255  seconds
\end{lstlisting}

Beide Varianten erhalten das selbe Ergebniss, allerdings sind die Prozesse kanpp 70 Sekunden schneller.
Ein gr��erer unterschied der beiden Varianten ist die Zeile \lstinline$if __name__ == ''__main__''$.
Diese Zeile ist im \hyperref[prozesse:lst:primzahen_prozesse]{Prozess-Beispiel} notwendig, da der
Programmcode innerhalb des \lstinline$if$-Statements ansonsten jedesmal ausgef�hrt wird, wenn das
Modul importiert wird.
Wird ein neuer Prozess gestartet, l�dt der neue Python-Interpreter das Modul ein zweites mal.
Sobald er nun die Codezeile erreicht, in der der neue Prozess gestartet wurde, wird ein dritter Prozesse
gestartet, dessen Python-Interpreter das Modul ein drittes mal l�dt.
Somit werden solange Prozesse erzeugt, bis das System keine Ressourcen mehr zur Verf�gung hat.
Durch angabe des \lstinline$if$-Statements ist es garantiert, dass der enthaltene Code nur einmal
ausgef�hrt wird, wenn das Programm gestartet wird.

\subsubsection{Prozess Objekte}
\label{prozesse:subsubsection:prozess_objekte}

Wie aus dem betrachteten \hyperref[prozesse:lst:primzahen_prozesse]{Prozess-Beispiel} erkenntlich ist,
ist die Verwendung der \lstinline$Process$-Klasse sehr stark an die der \lstinline$Thread$-Klasse angelehnt.
Genau genommen gibt es jede Methode von \lstinline$Thread$ auch in \lstinline$Process$.
Sogar die Konstruktoren sind identisch.
Der \lstinline$group$-Parameter des \lstinline$Process$-Konstruktors existert allerdings nur zur
Kompatibilit�t zum Konstruktor der \lstinline$Thread$-Klasse.
Der Aufruf von \lstinline$join()$ gibt immer \lstinline$Non$ zur�ck, auch falls der optionale Timeout
abgelaufen ist.
Um zu pr�fen, ob der entsprechende Prozess tats�chlich beendet wurde, ist �ber seinen
\lstinline$exitcode$
einsehbar.
Es ist zu bemerken, dass ein D�mon-Prozess keine weiteren Kindprozesse starten kann.
Sobald sich sein Elternprozess beendet, wird er terminiert.
Zus�ztlich bietet die \lstinline$Process$-Klasse noch weitere Attribute und Methoden an.
�ber das \lstinline$pid$-Attribut kann die ID des jeweiligen Prozesses abgefragt werden. 
Durch \lstinline$exitcode$ kann der entsprechende abgefragt werden, mit welchem Status sich der Prozess
beendet hat.
Wurde er noch nicht beendet, hat \lstinline$exitcode$ den Wert \lstinline$None$.
Tr�gt \lstinline$exitcode$ einen negativen Wert, so bedeutet das, dass der Prozess durch ein Signal
beendet wurde.
Ein Wert von \lstinline$-N$ entspricht dann dem Signal \lstinline$N$.
Beim Attribut \lstinline$authkey$ handelt es sich um einen Byte-String, welcher f�r gewisse
Authentifizierungen genutzt wird und standardm��ig den Wert des Elternprozesses �bernimmt.
Genauere Informationen zur Verwendung von \lstinline$authkey$ sind in \cite{pythondokuprozesse}
 zu finden.
Die beiden Methoden \lstinline$terminate()$ und \lstinline$kill()$ beenden den jeweiligen Prozess, nicht
aber dessen Kindprozesse.
Verwendet der Prozess \lstinline$Locks$, \lstinline$Semaphoren$, \lstinline$Queues$ oder \lstinline$Pipes$,
so kann ein Aufruf der beiden Methoden dazu f�hren, dass die verwendeten Objekte unnutzbar werden 
und andere Prozesse in einen Deadlock geraten.
Unter Windows wird zum Beenden der Prozesse \lstinline$TerminateProcess()$ aufgerufen.
Unter Unix-Systemen sendet \lstinline$terminate()$ das \lstinline$SIGTERM$ Signal, w�hrend
\lstinline$kill()$ das Signal \lstinline$SIGKKILL$ sendet.
Durch Aufruf der \lstinline$close()$-Methode werden alle Resourcen des jeweiligen Prozesses freigegeben,
falls er bereits beendet ist.
Andernfalls wird ein \lstinline$ValueError$ geworfen.
Nachdem \lstinline$close()$ erfolgreich zur�ckgekehrt ist, wird beim Zugriff auf die meisten Methoden 
und Attributen von \lstinline$Process$ ebenfalls ein \lstinline$ValueError$ geworfen.

\warning{
Die Methoden \lstinline$start()$, \lstinline$join()$, \lstinline$is_alive()$, \lstinline$terminate()$ und
\lstinline$exitcode$ sollten immer nur vom erzeugenden Prozess aufgerufen werden.
}

Das \lstinline$multiprocessing$-Modul unterst�tzt, je nach Betriebssystem, drei verschiedene Arten einen
Prozess zu starten.
Bei diesen drei Arten handelt es sich um \lstinline$spawn$, \lstinline$fork$ und \lstinline$forkserver$,
welche sich folgenderma�en unterscheiden:

\begin{enumerate}
\item \lstinline$spawn$: \newline
Diese Art der Prozesserzeugung starten einen neuen Python Interpreter. 
Es werden nur diejenigen Resourcen an den neuen Prozess vererbt, die zum Ausf�hren seiner 
\lstinline$run()$-Methode n�tig sind.
Im Vergleich zu den beiden anderen Varianten ist diese eher langsam.
Sie ist unter Unix und Windows Systemen verf�gbar und der Standard unter Windows.

\item \lstinline$fork$: \newline
Durch diese Startmethode wird ein Fork des aktuellen Python Interpreters durch den Aufruf von 
\lstinline$os.fork()$ erstellt.
Das hei�t, dass der erzeugte Kindprozess effektiv identisch zum Elternprozess ist. 
Alle Resourcen werden vom Elternprozess geerbt.
Werden erzeugen Multithreaded-Prozesse mit dieser Startmethode Kindprozesse, kann es sehr schnell
problematisch werden.
Unter Windows ist diese Startmethode nicht verf�gbar, unter Unix Systemen ist sie die Standardvariante.

\item \lstinline$forkserver$: \newline
Wurde beim Starten des Programms diese Variante zum Starten von Prozessen gew�hlt, wird ein Server 
gestartet.
Immer wenn ein neuer Prozess erzeugt werden soll, verbindet sich der erzeugende Prozess mit diesem
Server und fordert an, einen Fork erstellen zu lassen.
Hierbei werden nur die notwendigen Resourcen an den Kindprozess vererbt.
Da es sich bei dem Fork-Server um einen Single-Threaded-Prozess handelt, ist ein Aufruf von 
\lstinline$os.fork()$ unproblematisch.
Diese Variante wird nur von Unix Systemen unterst�tzt, die auch das �bergeben von Dateideskriptoren 
�ber Unix-Pipes unterst�tzen.
\end{enumerate}

Um eine der drei Startmethoden zu w�hlen kann die \lstinline$set_start_methode()$-Methode aufgerufen 
werden.
Sie sollte nur innerhalb von \lstinline$if __name__ ==$ \lstinline$''__main__''$ aufgerufen werden und nie mehrmals
in einem Programm.
Die \lstinline$spwan$ Startmethode wird also wie im
\hyperref[prozesse:lst:prozess_start_methode]{folgenden Beispiel} gezeigt ausgew�hlt.

\label{prozesse:lst:prozess_start_methode}
\lstinputlisting[language=Python,linerange={1-2,11-16}]{chapters/nebenlaufigkeit/src/prozess_start_methode.py}

Sollen mehrere Prozesse mit unterschiedlichen Startmethoden gestartet werden, so kann alternativ
\lstinline$get_context()$ aufgrufen werden, um ein \lstinline$Context$-Objekt zu erhalten.
Hierbei ist darauf zu achten, dass Objekte, welche mit einem \lstinline$Context$ erzeugt wurden, nicht
immer kompatibel mit Prozessen sind, welche mit einem anderen \lstinline$Context$ gestartet wurden.
So ist ein \lstinline$Lock$-Objekt, welches mit einem \lstinline$fork Context$ erzeugt wurde, nicht mit
Prozessen kompatibel, die mittels der \lstinline$spawn$ oder der \lstinline$forkserver$ Startmethode 
erzeugt wurde.
Wie ein Prozess mit einer bestimmten Startmethode durch ein \lstinline$Context$-Objekt erzeugt wird,
ist im folgenden \hyperref[prozesse:lst:prozess_start_methode_context]{Beispiel} gezeigt.

\label{prozesse:lst:prozess_start_methode_context}
\lstinputlisting[language=Python,linerange={1-2,17-22}]{chapters/nebenlaufigkeit/src/prozess_start_methode.py}

\uebung
\aufgabe{nebenlaufigkeit11}

H�ufig wird eine Aufgabe in kleinere Schritte unterteilt und dann an mehrere Arbeiterprozesse aufgeteilt.
Ein Beispiel hierf�r ist das vorherige \hyperref[prozesse:lst:primzahlen_prozesse]{Z�hlen von Primzahlen}.
F�r solche F�lle existiert in Python die \lstinline$Pool$-Klasse, welche das Auslagern von Aufgaben an 
Arbeiterprozesse erleichtert.
Wird ein neuer \lstinline$Pool$ erzeugt, kann seinem Konstruktor �ber den Parameter \lstinline$processes$
die Anzahl der Arbeiterthreads mitgegeben werden.
Werden die Parameter \lstinline$initializer$ und \lstinline$initargs$ angeben, so wird das Callable-Objekt,
welches an \lstinline$initializer$ �bergeben wurde von jedem der Arbeiterprozesse mit \lstinline$initargs$
als Parameter aufgerufen, wenn die Arbeiterprozesse gestartet werden.
Durch \lstinline$maxtasksperchild$ wird die maximale Anzahl an Aufgaben, welche die Arbeiterprozesse 
abarbeitern, bevor sie beendet und von einem neuen Arbeiterprozess ersetzt werden. 
Dies hat zur Folge, dass ungenutze Resourcen wieder freigegeben werden.
Der letzte Parameter ist \lstinline$context$.
Wird er angegeben, so werden die Arbeiterprozesse mit dem angegebenen \lstinline$Context$-Objekt 
erzeugt.
Um den erzeugten Arbeiterprozessen Aufgaben zu �bergeben, bietet \lstinline$Pool$ verschiedene
Methoden an.
Diese sollten immer nur von dem Prozess aufgerufen werden, der auch den \lstinline$Pool$ erzeugt hat.
Durch \lstinline$apply()$  wird von einem der Arbeiterprozesse das Callable-Objekt, welches �ber den
Parameter \lstinline$func$ angegeben wird, mit den Parametern, welche durch \lstinline$args$ angegeben
wurden, auf.
Es ist auch m�glich Schl�sselwort-Parameter �ber \lstinline$kwds$ anzugeben.
Der Aufruf von \lstinline$apply$ blockiert, bis die Aufgabe abgearbeitet wurde.
Die beiden Methoden \lstinline$map()$ und \lstinline$imap()$ nehmen ebenfalls ein Callable-Objekt
mit dem \lstinline$func$ Parameter entgegen.
Durch den Parameter \lstinline$iterable$ kann ein Iterable-Objekt �bergeben werden, welches 
aufgeteilt wird und den Arbeiterprozessen als seperate Aufgabe �bergeben wird.
Der Aufruf der beiden Methoden bockiert, bis die Aufgaben abgearbeitet wurden.
Bei sehr langen Iterable-Objekten ist \lstinline$imap()$ effizienter.
Ist die Reihenfolge der Ergebnisse irrelevant, kann auch \lstinline$imap_unordered()$ genutzt werden,
welche sich andernfalls genau wie \lstinline$imap()$ verh�lt.
Ben�tigt eine Aufgabe eine Parameterlist, welche gr��er 1 ist, kann \lstinline$starmap()$ angewendet 
werden.
Hierbei wird davon ausgegangen, dass die Elemente des Iterable-Objektes, welches an \lstinline$iterable$
�bergeben wird, ebenfalls Iterable-Objekte sind.
Diese werden als Parameter f�r das Callable-Objekte in \lstinline$func$ entpackt.
Wie die ersten 10 Quadratzahlen mit einem Pool aus 5 Prozessen berechnet werden k�nnen, wird im 
\hyperref[prozesse:lst:prozess_pool]{folgenden Beispiel} gezeigt.

\label{prozesse:lst:prozess_pool}
\lstinputlisting[language=Python,linerange={1-2,6-14}]{chapters/nebenlaufigkeit/src/prozess_pool.py}

Die Methoden \lstinline$apply_async()$, \lstinline$map_async()$ und \lstinline$startmap_async()$ verhalten
sich wie ihre normalen Varianten, aber sie blockieren nicht.
Sie geben ein \lstinline$AsyncResult$-Objekt zur�ck, �ber das das Ergebnis der Aufgaben erhalten 
werden kann, sobald es verf�gbar ist.
Die Methode \lstinline$ready()$ gibt an, ob die Aufgaben bereits abgearbeitet wurden.
Durch \lstinline$successful()$ kann erfahren werden, ob eine Exception w�hrend dem Bearbeiten der 
Aufgaben geworfen wurde. 
Wurden noch nicht alle Aufgaben vollst�ndig beabreitet, wird ein \lstinline$AssertionError$ geworfen.
Soll darauf gewartet werden, dass alle Aufgaben abgearbeitet wurden, kann \lstinline$wait()$ aufgerufen
werden.
Um das Ergebnis der Aufgaben zu erhalten, kann \lstinline$get()$ aufgerufen werden.
F�r \lstinline$wait()$ und \lstinline$get()$ kann ein optionaler Timeout angegeben werden.
Wurden alle Aufgaben an den \lstinline$Pool$ �bergeben, kann \lstinline$close()$ aufgerufen werden.
Nachdem diese Methode aufgerufen wurde, kann dem \lstinline$Pool$ keine neue Aufgabe mehr
�bergeben werden.
Sobald alle Aufgaben abgearbeitet wurden, werden die Arbeiterprozesse beendet.
Durch den Aufruf von \lstinline$terminate()$ werden die Arbeiterprozesse sofort beendet, ohne dass sie
ihre Aufgaben fertig bearbeiten.
Nachdem einer dieser beiden Methoden aufgerufen wurde, kann durch \lstinline$join()$ darauf gewartet
werden, dass alle Arbeiterprozesse beendet wurden.

\uebung
\aufgabe{nebenlaufigkeit12}
