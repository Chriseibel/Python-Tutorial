% !TeX root = ../pythonTutorial.tex
\chapter{Nebenl�ufigkeit}

Mit Nebenl�ufigkeit ist eine Eigenschaft von zwei oder mehr Aktivit�ten gemeint. Eine beliebige Anzahl an Aktivit�ten wird als nebenl�ufig bezeichnet, wenn die Reihenfolge der Ausf�hrung der einzelnen Aktivit�ten irrelevant f�r das Ergebnis ist. Mit anderen Worten ist es egal, ob zu erst die eine Aktivit�t und dann die andere Aktivit�t ausgef�hrt wird, oder umgekehrt, oder sogar beide zeitgleich. Hierbei ist auf den Unterschied von zwischen Nebenl�ufigkeit und Parallelit�t zu achten. W�hrend Nebenl�ufigkeit eine Eigenschaft darstellt, ist Parallelit�t eine m�gliche Herangehensweise an das Ausf�hren von nebenl�ufigen Aktivit�ten.

Im folgenden Kapitel wird beschrieben, wie in Python durch Threads und Prozesse nebenl�ufige Programmierung realisiert werden kann. Python unterscheidet sich in Bezug auf Parallelit�t stark von anderen Programmiersprachen. In Abschnitt \ref{parallelit�t_in_python} wird auf diesen Unterschied eingegangen. Anschlie�end wird in Abschnitt \ref{threads} beschrieben, wie in Python Nebenl�ufigkeit durch Threads realisiert werden kann. Neben der Erzeugung von Threads werden Synchronisations- und Steuerungsmechanismen besprochen. Nachdem nun Nebenl�ufigkeit durch Threads erreicht wurde, wird sich in Abschnitt\ref{prozesse} auf Prozesse bezogen.

\input{chapters/sections/parallelit�t_in_python.tex}

% !TeX root = ../../pythonTutorial.tex
\section{Threads}

\label{threads}

% !TeX root = ../../pythonTutorial.tex
\label{prozesse:section:prozesse}
\section{Prozesse}