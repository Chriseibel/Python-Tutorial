% !TeX root = ../../pythonTutorial.tex
In diesem Kapitel gehen wir auf die Nutzung von Funktionen in Python ein.
Eine Funktion bildet einen Code-Block ab, also eine Sequenz von Befehlen, die eine bestimmte \textbf{Funktion} erf�llt. 


Dieser Code-Block wird mit dem Schl�sselwort \lstinline$def$ gestartet, gefolgt vom Namen der Funktion, anschlie�end von Klammern (welche Input-Parameter beinhalten k�nnen). Zum Schluss der Funktionsdefinition folgt noch ein Doppelpunkt, nach diesem kommt die Befehlssequenz.
Soll die Funktion Werte zur�ckliefern, dann steht am Ende der Sequenz das \lstinline$return$-Schl�sselwort. Achtung: eine Funktion liefert immer einen Wert zur�ck. Wird keiner angegeben, so wird der Wert \lstinline$None$ als R�ckgabewert festgelegt.

\begin{lstlisting}[language=Python, label=einleitungFunktionen:lst:funcDef]
# Definition einer Funktion in Python

def funktionsname (parameter):
  ...
  return rueckgabewert
\end{lstlisting}


