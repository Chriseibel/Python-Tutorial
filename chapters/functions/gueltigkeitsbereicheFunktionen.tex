% !TeX root = ../../pythonTutorial.tex
\section{G�ltigkeitsbereich von Variablen}

Je nachdem, wie und wo Funktionen definiert sind, k�nnen diese nach einem Aufruf zu anderen Ergebnissen f�hren. Zur Verdeutlichung folgt ein einfaches Beispiel (\ref{gueltigkeitsbereicheFunktionen:lst:simpleScope}).

\begin{lstlisting}[caption=Code-Beispiel zum G�ltigkeitsbereich, label=gueltigkeitsbereicheFunktionen:lst:simpleScope]
def myFunction():
  # lokaler G�ltigkeitsereich der Funktion
  a = 1
  print('myFunction:', a)

# globaler G�ltigkeitsbereich
a = 0
myFunction()
print('global:', a)
\end{lstlisting}

Wenn wir diesen Code ausf�hren, sehen wir folgenden Output:

myFunction: 1
\\*
global: 0

In beiden Bereichen benutzen wir die Variable \lstinline$a$.
Die Funktion wird nach dem Initialisieren der Variable aufgerufen,
warum erhalten wir also zwei unterschiedliche Werte?

Der Grund daf�r ist, weil es sich eben nicht um die gleiche Variable handelt, denn beide wurden in verschiedenen G�ltigkeitsbereichen definiert.
W�rden wir die lokale Zuweisung auslassen, w�rde zweimal der Wert 0 gedruckt werden.
Python sucht nach dem n�chstm�glichen G�ltigkeitsbereich: lokal, umschlie�end, global und \lstinline$built-in$.

Nun ein Beispiel, mit verschachtelten Funktionen:

\begin{lstlisting}[caption=Verschachtelte Funktionen, label=gueltigkeitsbereicheFunktionen:lst:enclosingScope]
def enclosing():
  a = 1

  def innerFunction():
    a = 2
    print('innerste:', a)

  innerFunction()
  print('umschlie�end:', a)

a = 0
enclosing()
print('global:', a)
\end{lstlisting}

Nach dem Ausdrucken erhalten wir:

innerste: 2
\\*
umschlie�end: 1
\\*
global: 0


\subsection{Statements zu G�ltigkeitsbereichen - global und nonlocal}

Nicht nur durch die Positionen werden G�ltigkeitsbereiche definiert, durch die Schl�sselw�rter \lstinline$global$ und \lstinline$nonlocal$ k�nnen wir den G�ltigkeitsbereich bestimmen.

Durch nonlocal wird eine Variable auf die n�chst umschlie�ende Definition festgelegt (\ref{gueltigkeitsbereicheFunktionen:lst:nonlocalStatement}). 

\begin{lstlisting}[caption=Nonlocal Statement, label=gueltigkeitsbereicheFunktionen:lst:nonlocalStatement]
def enclosing():
  a = 1

  def innerFunction():
    \textbf{nonlocal a}
    a = 2
    print('innerste:', a)

  innerFunction()
  print('umschlie�end:', a)

a = 0
enclosing()
print('global:', a)
\end{lstlisting}

Nach dem Ausdrucken erhalten wir:

innerste: 2
\\*
umschlie�end: 2
\\*
global: 0

Achtung: W�rde in der enclosing-Funktion \lstinline$a$ auf nonlocal gesetzt, dann k�me es zu einer \lstinline$Exception$, da die n�chste Ebene global ist.

Das gleiche k�nnen wir mit dem globalen G�ltigkeitsbereich machen, wie in \ref{gueltigkeitsbereicheFunktionen:lst:globalStatement} gezeigt.

\begin{lstlisting}[caption=Global Statement, label=gueltigkeitsbereicheFunktionen:lst:globalStatement]
def enclosing():
  a = 1

  def inner():
    \textbf{global a}
    a = 2
    print('innerste:', a)

  innereFunction()
  print('umschlie�end:', a)

a = 0
enclosing()
print('global:', a)
\end{lstlisting}

Nach dem Ausdrucken erhalten wir:

innerste: 2
\\*
umschlie�end: 1
\\*
global: 2


W�hrend nonlocal nur den n�chst umschlie�enden G�ltigkeitsbereich w�hlt -  in welcher die Variable deklariert wurde - greift global immer auf den globalen G�ltigkeitsbereich zu.