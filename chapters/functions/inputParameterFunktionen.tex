% !TeX root = ../../pythonTutorial.tex
\section{Input-Parameter}

In diesem Abschnitt sehen wir uns die verschiedenen Arten von �bergabeparametern genauer an, sowie die M�glichkeiten, diese an eine Funktion zu �bergeben.

\begin{lstlisting}[caption=�bergabe eines primitiven Datentypes, label=inputParameterFunktionen:lst:primitiveParameter]
def myFunction(x):
  x = 1
x = 2
myFunction(x)
print(x)
\end{lstlisting}

Mittels Codes (\ref{inputParameterFunktionen:lst:primitiveParameter}) wird der Wert '2' ausgedruckt. Aus dem vorherigen Abschnitt �ber G�ltigkeitsbereiche wissen wir, dass es sich hierbei - trotz gleichem Namen - um zwei verschiedene Variablen handelt. Eine Integer-Variable im lokalen und eine im globalen G�ltigkeitsbereich. Dies gilt jedoch nur f�r primitive Datentypen, f�r nicht-primitive Datentypen verh�lt es sich anders.

\begin{lstlisting}[caption=�bergabe eines nicht-primitiven Datentyps, label=inputParameterFunktionen:lst:nonPrimitiveParameter]
def myFunction(x)
  x[0] = 100
x = [0,1,2]
myFunction(x)
print(x)
\end{lstlisting}

Da es sich hier (\ref{inputParameterFunktionen:lst:nonPrimitiveParameter}) nicht um einen primitiven Datentyp handelt, stellt der Input-Parameter eine Referenz dar, womit der Wert von x auch in der Methode ge�ndert wird. Beim Ausdrucken erhalten wir: \lstinline$[100,1,2]$.

Es ist wichtig, sich dieses Verhalten zu merken, da beim objekt-orientierten Programmieren sonst Fehler auftreten.


\subsection{Arten von Input-Parametern}

In folgendem Unterabschnitt gehen wir auf verschiedene Arten von Input-Parametern ein.

\subsubsection{Positionale Parameter}

Diese Art von Parametern d�rfte bereits bekannt sein. Es handelt sich um eine endliche Anzahl von Parametern, die man von links nach rechts liest.

\begin{lstlisting}[caption=Funktion mit positionalen Parameter, label=inputParameterFunktionen:lst:positionalParameter]
def myFunction(x,y,z):
  ...
  myFunction(1,2,3)
\end{lstlisting}


\subsubsection{Schl�sselwort-Parameter}

Bei Schl�sselwort-Parametern ist die Reihenfolge der Parameter nicht elementar, da die Variablen �ber den Namens-Wert �bergeben werden.

\warning{Bei der Verwendung von positionalen und Schl�sselwort-Parametern, m�ssen positionale Parameter immer links der Schl�sselwort-Parameter stehen.}

\begin{lstlisting}[caption=Funktion mit Schl�sselwort-Parametern, label=inputParameterFunktionen:lst:keywordParameter]
def myFunction(x,y,z):
  ...
  myFunction(x = 1, z = 3, y = 2)
\end{lstlisting}


\subsubsection{Standard-Parameter}

Standard-Parameter folgen einer �hnlichen Syntax wie Schl�sselwort-Parameter, sie werden jedoch in der Funktionsdefinition festgelegt und nicht beim Aufruf.
Zu beachten ist, dass beim Verwenden von positionalen und Standard-Parametern alle Standard-Parameter \textit{nach} den positionalen stehen m�ssen.

\begin{lstlisting}[caption=Funktion mit Standard-Parametern, label=inputParameterFunktionen:lst:standardParameter]
def myFunction(x, y = 10, z = 20):
  print(x, y, z)

myFunction(0)                       # 0 10 20
myFunction(y = 20, x = 10, z = 30)  # 10 20 30
myFunction(0, z = 1)                # 0 10 1
\end{lstlisting}

\subsection{Rest-Parameter}

Rest-Parameter erm�glichen quasi eine unendliche Anzahl an �bergabe-Parametern.
M�chte ein Nutzer beispielsweise den Durchschnitt mehrerer Zahlen wissen, so k�nnte die
Funktion wie folgt aussehen:

\begin{lstlisting}[caption=Funktion mit Rest-Parametern, label=inputParameterFunktionen:lst:restParameter]
def average(\textbf{*numbers}):
  value = 1
  for number in numbers:
    value = value * number
    
  value = value / len(numbers) # durch die Anzahl an Parametern 
  return value

result = average(5, 5, 10, 8, 2)
print(result) # druckt das Ergebnis 6
\end{lstlisting}

Die multiplen Parameter werden als Liste gehandhabt.
Bei der Definition von Rest-Parametern in einer Funktion ist es wichtig, dass sie als letztes Argument stehen.

