% !TeX root = ../../pythonTutorial.tex
\section{Konsolenausgabe mit print()}
\label{printInOut}

Die \emph{print()}-Funktion hat vielerlei Nutzen und wird entsprechend oft verwendet. Daher ist es definitiv sinnvoll, sich mit ihr vertraut zu machen.

\begin{lstlisting}[language=Python, label=printInOut:lst:printDefault]
# die Print-Methode

print(value1, value2, ..., sep=" ", 
    end="\n", file=sys.stdout, flush= false))
\end{lstlisting}

In Listing (\ref{printInOut:lst:printDefault}) sind die Parameter von \emph{print()} zu sehen. Die auszugebenden Werte stehen an erster Stelle (value1, value2, ...), hierbei handelt es sich um eine beliebige Anzahl an Werten. Der Parameter \emph{sep} bildet den Seperator zwischen den Werten und hat als Standardwert das Leerzeichen.
\randnotiz{Seperator}
In Listing \ref{printInOut:lst:printSeperator} k�nnen wir sehen, dass wir durch Angabe eines Seperators eine bessere Lesbarkeit erreichen k�nnen.

\begin{lstlisting}[language=Python, label=printInOut:lst:printSeperator]
# die Print-Methode mit Seperator

print(1,2,3)
# Ausgabe: 1 2 3

print(1,2,3, sep=" | ")
# Ausgabe: 1 | 2 | 3
\end{lstlisting}

Nach dem Seperator folgt der \emph{end}-Parameter, dieser f�gt standardgem�� einen Zeilenumbruch (\textbackslash n) an das Ende der Ausgabe.
\randnotiz{End-Angabe}

\begin{lstlisting}[language=Python, label=printInOut:lst:printEnd]
# die Print-Methode mit End-Angabe

print("Satz mit Zeilenumbruch")
print("N�chster Satz")
# Ausgabe:
# Satz mit Zeilenumbruch
# N�chster Satz

print("Satz mit Punkt und Leerzeichen." , end=". ")
print("N�chster Satz)
# Ausgabe: 
# Satz mit Punkt und Leerzeichen. N�chster Satz
\end{lstlisting}

Der \emph{file}-Parameter bestimmt den Datenstrom (\emph{Stream}) f�r den Output. In \emph{sys.stdout} steht f�r die Konsole. 
\randnotiz{Output-Stream}
M�chten wir den Output bspw. in eine Textdatei schreiben, dann k�nnen wir diese als Ziel des Datenstroms festlegen.

\begin{lstlisting}[language=Python, label=printInOut:lst:printFile]
# die Print-Methode mit Ausgabe in Textdatei

# open() -> festlegen einer Textdatei als Stream
# "w" steht f�r 'write' und gibt an,
# dass wir etwas in die Datei schreiben m�chten

txtFile = open(beispielText.txt, "w")
print("Hello, World.", file="txtFile")
txtFile.close()
# close() schlie�t den Stream
\end{lstlisting}



