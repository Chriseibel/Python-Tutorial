% !TeX root = ../../pythonTutorial.tex
\section{Konsoleneingabe mit input()}
\label{inputInOut}

Mit Hilfe von \emph{input()} erlauben wir dem Nutzer Eingaben �ber die Konsole. Somit erhalten wir den ersten Grad an Interaktion zwischen Nutzer und Programm.

\begin{lstlisting}[language=Python, label=inputInOut:lst:inputDefault]
# die input-Methode

input(prompt)
\end{lstlisting}

Sobald \emph{input()} aufgerufen wird, wartet das Programm mit dem weiteren Ablauf, bis der Nutzer seine Eingabe mit der Eingabetaste best�tigt.
Die \emph{input()}-Methode liefert den eingegebenen Wert als String zur�ck.
Damit der Nutzer wei�, was er denn eingeben muss, bietet \emph{input()} den optionalen Standard-Parameter \emph{prompt} an - hierbei handelt es sich um einen leeren String.
Geben wir \emph{prompt} nun einen Wert, wird dieser dem Nutzer f�r die Eingabe angezeigt.

\begin{lstlisting}[language=Python, label=printInOut:lst:inputPrompt]
# die input-Methode mit prompt-Angabe

userName = input("Geben Sie Ihren Namen ein.")
print("Hallo, " + userName)
\end{lstlisting}

\warning{Der Eingabewert des Nutzer liefert immer einen String zur�ck.
Bei gew�nschtem Datentyp, muss \emph{gecasted} werden!}

Bei primitiven Datentypen ist das Umwandeln recht einfach. Bei nicht-primitiven kann es jedoch zu �berraschungen kommen.

\begin{lstlisting}[language=Python, label=printInOut:lst:inputCast]
# die input-Methode mit Typ-Umwandlung

summe = int(input("2 + 3 = "))
print(summe, type(summe), sep=" - ")
# Eingabe: 5
# Ausgabe: 5 - <class 'int'>

geoKoerper = list(input("Geben Sie einige" +
			"geometrische K�rper an"))
print(geoKoerper)
# Eingabe: ["Dreieck", "Viereck"]
# Ausgabe: [' ', '[', '"', 'D', 'r', 'e', 
	'i', 'e', 'c', 'k', '"', ',', 
	' ', '"', 'V', 'i', 'e', 'r', 
	'e', 'c', 'k', '"', ']']
			
print(type(geoKoerper[0]))
# Ausgabe: <class 'str'>

\end{lstlisting}

Python wandelt den String in eine Liste um, jedoch nimmt er jedes einzelne Zeichen der Eingabe als Listenelement. Dies kann durchaus n�tzlich sein, verfehlt hierbei aber das Ziel. Um die geometrischen K�rper als Elemente zu erhalten, nutzen wir die eval()-Funktion.
\randnotiz{eval()}
Hierbei wird die Eingabe interpretiert und der entsprechende Datentyp zur�ckgeliefert (Evaluierung).

\tip{eval() funktioniert auch bei anderen Collections!}

\begin{lstlisting}[language=Python, label=printInOut:lst:inputEval]
# die input-Methode mit eval()

geoKoerper = eval(input("Geometrische K�rper: "))
print(geoKoerper)
# Eingabe: ["Dreieck", "Viereck"]
# Ausgabe: ['Dreieck', 'Viereck']
\end{lstlisting}



