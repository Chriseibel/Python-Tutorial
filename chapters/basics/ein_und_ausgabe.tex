% !TeX root = ../../pythonTutorial.tex
\section{Ein- und Ausgabe}
\label{ein_und_ausgabe}

Das Beschreiben und Einlesen einer Datei in Python kann wie folgt vollzogen werden:

\subsection{Datei beschreiben}
\label{ein_und_ausgabe:datei_beschreiben}

\lstinputlisting[language=Python]{chapters/basics/src/ein_und_ausgabe/DateiBeschreiben.py}

Um in Python 3 eine Datei zu erzeugen und anschlie�end zu beschreiben verwendet man die Funktion \lstinline$open()$.\\
In unserem Beispiel �bergeben wir der Funktion \lstinline$open()$ zwei Argumente. Bei dem Ersten handelt es sich hierbei um eine absolute Pfadangabe, die als String-Objekt �bergeben wird. Damit wird die Datei \lstinline$Daten.txt$ in dem aktuellen Verzeichnis, wo sich auch die \lstinline$DateiBeschreiben.py$ befindet, erzeugt. Der nachfolgende Parameter \lstinline$w$ (write) dient als Modus-Auswahl und sorgt daf�r, dass die Datei beschrieben werden kann.
Sollte hierbei eine Datei mit demselben Namen bereits existieren, wird diese �berschrieben.
Anschlie�end bekommen wir eine Referenz auf das Datei-Objekt.
\\Hinweis: Zum Beschreiben der Dateien wird grunds�tzlich ein Schreibrecht ben�tigt.

Mit der \lstinline$write$-Methode k�nnen wir nun gezielt unsere Daten in die Datei schreiben.
Abschlie�end muss noch die \lstinline$close$-Funktion aufgerufen werden. Diese sorgt daf�r, dass die Datei nach dem Schreibvorgang ordnungsgem�� geschlossen wird.


\subsection{Datei einlesen}
\label{ein_und_ausgabe:datei_einlesen}

\lstinputlisting[language=Python]{chapters/basics/src/ein_und_ausgabe/DateiEinlesen.py}

F�r das Einlesen einer Datei wird hierbei ebenfalls die \lstinline$open$-Funktion verwendet. Dabei wird
als zweiter Parameter ein \lstinline$r$ f�r \lstinline$read$ �bergeben.
Das Auslesen der Datei kann mithilfe einer Schleife realisiert werden.

Falls man zeilenweise einlesen m�chte, kann man die \lstinline$readlines$-Methode wie folgt verwenden:

\begin{lstlisting}[label=ein_und_ausgabe:readlines]
datei.readlines()
\end{lstlisting}
