% !TeX root = ../../pythonTutorial.tex

\section{Kontrollstrukturen}
\label{kontrollstrukturen:sec:Kontrollstrukturen}

Die Kontrollstrukturen in Python haben einen formalen Unterschied zu Java oder C++, funktional allerdings sind sie identisch. In Python werden keine geschweiften Klammern genutzt, um die Bl�cke der einzelnen Abfragen abzugrenzen. Dazu gen�gt das Einr�cken der Anweisung. Dies gilt sowohl f�r Bedingungen und Conditional Expressions, als auch f�r Schleifen. Im Folgenden schauen wir uns die einzelnen Strukturen im Detail und mit Beispielen an.

\subsection{If-then-else}
\label{kontrollstrukturen:sec:ifthenelse}

Die if-then-else-Struktur erm�glicht es, wie wir es bereits kennen, simple wenn-dann Abfragen zu t�tigen.\\
Mehrere Abfragem�glichkeiten werden mit elif markiert. Vergleich hierzu Listing \ref{kontrollstrukturen:lst:refif}.



\begin{lstlisting}[label=kontrollstrukturen:lst:refif]
# If-then-else
if statement1:
	print("Fall 1")
elif statement2:
	print("Fall 2")
else:
	print("Fall 3")
\end{lstlisting}

\textbf{Conditional Expressions}

Die Conditional Expressions (engl. bedingte Ausdr�cke) stellen eine kompaktere Schreibweise als if-then-else-Bedinungen dar. Ein Beispiel ist in Listing \ref{kontrollstrukturen:lst:refcond} zu finden. 

\begin{lstlisting}[label=kontrollstrukturen:lst:refcond]
# Conditional Expressions
# Klassisches If-Else
if wort == "start":
	x = "los"
else:
	x = halt"
	
# If-Else als Conditional Expression
x = ("los" if wort == "start" else "halt")

\end{lstlisting}


\subsection{Schleifen}
\label{kontrollstrukturen:sec:Schleifen}

Python hat sowohl bedingte, als auch Z�hler-Schleifen, welche wir uns beide im Folgenden genauer ansehen werden (vgl. Listing \ref{kontrollstrukture:lst:refwhile} und \ref{kontrollstrukture:lst:reffor}). Schleifen bestehen aus einer Anweisung und einem Kontrollblock, welcher solange durchlaufen wird, bis die Anweisung oder ein Abbruchkriterium erf�llt wurde. Schleifen, die niemals ein Abbruchkriterium erf�llen und so endlos durchlaufen werden, hei�en Endlosschleifen. Diese f�hren dazu, dass der Interpreter irgendwann aufgibt und abbricht.

\begin{lstlisting}[label=kontrollstrukturen:lst:refwhile]
# While-Schleife
while Bedingung:
	Anweisungsblock
	if Bedingung:
		Anweisungsblock
		continue
	if Bedingung:
		Anweisungsblock
		break
	Anweisungsblock

\end{lstlisting}


\begin{lstlisting}[label=kontrollstrukturen:lst:reffor]
# For-Schleife
for Variable in Objekt (von, bis, Variablenver�nderung):
	Anweisungsblock
	if Bedingung:
		Anweisungsblock
		continue
	Anweisungsblock
	if Bedingung:
		Anweisungsblock
		break
	Anweisungsblock

\end{lstlisting}

Die while-Schleife erinnert stark an die Verwendung in anderen Programmiersprachen. Wird jedoch die for-Schleife betrachtet, fallen einige Unterschiede bei der Beschreibung der Anweisung auf. Die Variable wird einmalig am Anfang der Anweisung definiert. Anschlie�end wird die range, also die Grenzen von wo bis wo die Variable in der Schleife durchgegangen werden soll. Die beiden Grenzen werden mit einem Komma getrennt. F�gt man eine dritte Zahl dahinter ein, dient diese dazu die Variable bei jedem Durchgang der Schleife zu ver�ndern. L�sst man diese Zahl weg, so wird die Variable bei jedem Schleifendurchlauf um eins erh�ht. Schreibt man beispielsweise eine 3, wird die Variable jedes Mal, wenn die Schleife erneut durchgegangen wird um 3 erh�ht.

Zur Veranschaulichung finden Sie im Folgenden zwei Beispiele zur Verwendung einer while- und einer for-Schleife. Beide Schleifen bilden die Summe der Zahlen von 1 bis 10.

\begin{lstlisting}[label=kontrollstrukturen:lst:bspwhile]
# Beispiel als While-Schleife

n = 10
s = 0
i = 1

while i <= n:
   	s = s + i
	i = i + 1

print ("Summe:", s)

# Ausgabe: "Summe: 55"
\end{lstlisting}


\begin{lstlisting}[label=kontrollstrukturen:lst:bspfor]
# Beispiel als For-Schleife

s = 0

for i in range (0,11):
    s = s + i
    
print ("Summe: ", s)

# Ausgabe: "Summe: 55"
\end{lstlisting}

\subsection{Ausdr�cke und Operatoren}
\label{grundlagen:sec:Ausdr�ckeundOperationen}

Die meisten Operatoren f�r Zahlenwerte sind in Python �hnlich wie bei anderen Programmiersprachen. Im folgenden wird eine �bersicht gegeben.

\begin{table}[h]

\begin{tabular}{|p{0.15\textwidth}|p{0.5\textwidth}|p{0.25\textwidth}|}
\hline
\multicolumn{1}{|c|}{\textbf{Operator}} & \multicolumn{1}{c|}{\textbf{Bezeichnung}} & \multicolumn{1}{c|}{\textbf{Beispiel}} \\ \hline
\hline
+, - & Addition, Subtraktion & 4 - 3 \\ \hline
*, \% & Multiplikation, Rest & 24 \% 5 \newline Ergebnis: 4 \\ \hline
/ & Division & 10 / 3 \newline Ergebnis: 3.33333333333335 \\ \hline
// & Ganzzahldivision & 10 // 3 \newline Ergebnis: 3 \\ \hline
+x, -x & Vorzeichen & -5 \\ \hline
** & Exponentiation & 2 ** 4 \newline Ergebnis: 16 \\ \hline 
or, and, not & Boolsches Oder / Und / Nicht & (a or b) and c \\ \hline
in & Element von & 1 in [1,2,3]  \\ \hline
<, <=, >, >=, !=, == & Vergleichsoperatoren & 4 <= 5 \\ \hline
\end{tabular}
\caption{Ausdr�cke und  Operatoren}

\end{table}


