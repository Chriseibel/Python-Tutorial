% !TeX root = ../../pythonTutorial.tex

\section{Elementare Datentypen}
\label{basicdatatypes:sec:ElementareDatentypen}

�hnlich wie bei Java und C oder C++ gibt es auch in Python Variablen. Allerdings gibt es dabei immense Unterschiede zu den anderen Programmiersprachen, weshalb sich ein genauerer Blick auf die einzelnen Datentypen in jedem Fall lohnt. Bei vielen bekannten Sprachen wird einer Variablen ein bestimmter Datentyp zugeordnet (deklariert). Der Datentyp kann darauf folgend zur Laufzeit nicht wieder ge�ndert werden, der Wert innerhalb des Datentyps allerdings schon. So lassen sich in eine Variable des Typ Integer beispielsweise keine String-Werte speichern. In Python hingegen ist dies ohne weiteres m�glich. Hier wird g�nzlich auf eine explizite Typdeklaration verzichtet. Zeigt eine Variable beispielsweise auf eine ganze Zahl, so wird diese als ein Objekt vom Typ Integer interpretiert. Allerdings ist es m�glich, diese im n�chsten Schritt einfach auf ein String-Objekt zeigen zu lassen. Dies ist in Python m�glich, weil eine Variable ein Objekt lediglich referenziert und dadurch keinem Typ zugewiesen wird.\\
Betrachten wir nun die Datentypen etwas genauer.

\subsection{Zahlenoperatoren}
\label{basicdatatypes:sec:Zahlenoperatoren}

Da in Python auf Typdeklaration verzichtet wird, muss dies beim Anlegen von Variablen nicht ber�cksichtigt werden. Wird eine ganze Zahl (Integer) ben�tigt, kann diese, falls n�tig, auch in eine Gleitkommazahl (float) umgewandelt werden, ohne viel am Code zu �ndern. Python deklariert im Hintergrund selbst und spart so unn�tige Komplexit�ten und Fehlerquellen. (Beispiel \ref{refzahl})

\begin{lstlisting}[label=refzahl]
# Zahlenoperatoren
i = 42
type(i)
// Ausgabe: <class 'int'>
i = 42.22
type(i)
// Ausgabe: <class 'float'>
\end{lstlisting}

\textbf{Boolean}

Boolean gibt an, ob ein Statement \textit{true} oder \textit{false} ist. Dadurch lassen sich Fallunterscheidungen oder Abfragen erm�glichen. (Beispiel in Listing \ref{basicDatatypes:lst:refbool})

\begin{lstlisting}[label=basicDatatypes:lst:refbool]
# Boolean
i = True
i
// Ausgabe: True

\end{lstlisting}

\textbf{String}

Der String ist eine Zeichenkette, also eine Aneinanderreihung von verschiedenen Zeichen. Dazu z�hlen W�rter, aber auch beispielsweise Hexadezimal-Codes oder E-Mail Adressen.

Wie in den meisten objektorientierten Programmiersprachen lassen sich auch in Python die einzelnen Zeichen eines Strings abrufen, indem der dazugeh�rige Index abgefragt wird.

Wie in Listing \ref{basicDatatypes:lst:refstring} kann die L�nge des gesamten Strings durch einfache Abfrage angezeigt werden. 

\begin{lstlisting}[label=basicDatatypes:lst:refstring]
# Strings
i = "Python"
print (i)
// Ausgabe: Python

print(i[0])
// Ausgabe: P

print(len(i))
// Ausgabe: 6

\end{lstlisting}

\subsection{ENUMs}
\label{basicdatatypes:sec:Enums}

Enums dienen in den objektorientierten Programmiersprachen zur Aufz�hlung von Ausdr�cken einer endlichen Menge. So werden zum Beispiel Jahreszeiten, Monate oder Farben oft als Enums umgesetzt (vgl. Listing \ref{refenum}). 


\begin{lstlisting}[label=refenum]
# Enums
from enum import Enum
class Color(Enum):
	RED = 1
	GREEN = 2
	BLUE = 3

\end{lstlisting}

\subsection{NULL oder NONE}
\label{basicdatatypes:sec:NullNone}
Das Schl�sselwort \textit{NULL} wird in vielen Programmiersprachen genutzt. Die Idee dahinter ist einer Variable ein neutrales Verhalten zu geben. Das �quivalent zu \textit{NULL} in Python ist \textit{NONE}. Der Vorteil ist, dass \textit{NONE} exakt der Aufgabe des Schl�sselworts entspricht. Ein Anwendungsfall f�r \textit{NONE} w�re beispielsweise um zu �berpr�fen, ob die Verbindung zu einer Datenbank aufgebaut werden konnte oder nicht (Siehe Beispiel \ref{refnone}).

\begin{lstlisting}[label=refnone]
# NULL oder NONE
database_connection = None

try:
    database = MyDatabase(host, user, password, database)
    database_connection = database.connect()
except DatabaseException:
    pass
 
if database_connection is None:
// Solange die Variable "NONE", keine Verbindung aufgebaut					
    print('The database could not connect')
else:
    print('The database could connect')
    
\end{lstlisting}

\subsection{Referenz, Identit�t und Kopie}
\label{basicdatatypes:sec:Referenzen}

Wie bereits erw�hnt wurde, wird in Python eine Variable keinem Typ zugewiesen. Zeigt eine Variable jedoch st�ndig auf ein neues Objekt, sind Verwechslungen innerhalb des Codes m�glich. Um dies zu vermeiden bietet sich die Identit�tsfunktion id() an. Diese hilft uns dabei, die verschiedenen Instanzen voneinander zu unterscheiden. Jede Instanz hat dabei unabh�ngig von ihrem Wert und ihrem Typ eine eindeutige Identit�t. \\

Dies ist in Python m�glich, weil eine Variable ein Objekt lediglich referenziert und dadurch keinem Typ zugewiesen wird.

\newpage

\uebung
\aufgabe{DatatypesAufgabe1}
\aufgabe{ifelseAufgabe1}
\aufgabe{KontrollstrukturenAufgabe1}