% !TeX root = ../../pythonTutorial.tex
\section{Testing}
\label{tests:sec:Testing}
Zum Testen unter Python gibt es mehrere M�glichkeiten.
Zum einen das Modul \lstinline$doctest$, welches als interaktive Dokumentation Testmuster bereitstellt.
Zum anderen gibt es das Modul \lstinline$unittest$.
Dieses Modul implementiert in Python die unter anderem aus JUnit bekannte Regressionstesten.
\subsection{doctest}
\label{tests:sec:DocTest}
Das Modul \lstinline$doctest$ erm�glicht Tests in Form von Kommentaren direkt anzuh�ngen.
Dies ist sowohl innerhalb einer Funktion als auch au�erhalb m�glich.
Hierbei werden die Tests in Kommentarbl�cken (\lstinline$"""$) durch die Zeichenfolge \lstinline$>>>$ eingeleitet.
Danach wird die Funktion, die getestet werden soll mit den gew�nschten Testparametern aufgerufen.
Diesem folgt dann entweder das Ergebnis oder eine Fehlerbehandlung im Falle einer zu pr�fenden Exception.

\lstinputlisting[language=Python,lastline=37, label={tests:lst:SimpleDocTest}]{chapters/basics/src/tests/simpleDocTest.py}

Beispiel \ref{tests:lst:SimpleDocTest} zeigt, wie \lstinline$doctests$ in der Praxis Anwendung finden.
Die Tests werden, sobald der Code ausgef�hrt wird mit ausgef�hrt. 
Dadurch das die Tests in der Funktion stehen besteht ein direkter Bezug zwischen Tests und dem zu testenden Code.
Solle kein Fehler auftreten, erscheint keine Ausgabe.
Ist trotzdem eine Ausgabe erw�nscht, ist diese mit dem Parameter \lstinline$-v$ (verbose) aktivierbar.

Im nachfolgenden Beispiel \ref{tests:lst:SimpleDocTestPrintSuccess} sind die Tests aus dem oben gezeigten Beispiel \ref{tests:lst:SimpleDocTest}, zu sehen.
Der Ablauf ist jedes Mal gleich.
Zuerst wird der Test, mit den Versuchsparametern aufgef�hrt und im Anschluss der Erwartungswert gezeigt.
Sollte der R�ckgabewert der Funktion dem Erwartungswert entsprechen, wird der Test mit OK beendet.
Gleiches gilt beim Testen auf Exceptions.
Zum Schluss werden die Ergebnisse in einer Auflistung zusammengefasst und nach Zugeh�rigkeit gruppiert.
Hierbei handelt es sich um einen freien Test und 3 innerhalb der Funktion.
Danach gibt es noch eine weitere Zusammenfassung, die die Tests nach Erfolg und Misserfolg gruppiert.
Dies soll dem Anwender erm�glichen, alle Tests mit einem Blick zu erfassen.

\lstinputlisting[language=python,linerange={1-3,40-69}, label={tests:lst:SimpleDocTestPrintSuccess}]{chapters/basics/src/tests/simpleDocTest.py}

Um auch denn Fehlerfall zu betrachten, habe ich die R�ckgabe von \lstinline$x*y$ auf \lstinline$x+y$ ge�ndert.
Dies sorgte sofort f�r zwei Fehler beim Start des Programms.
Die einzelnen Fehler werden durch Balken getrennt.
Im Unterschied zum Erfolgsfall wird hier die Stelle angegeben, an der der Test steht.
Zus�tzlich dazu wird der zur�ckgegebene Wert angezeigt.
Am ende erfolgt wieder eine Zusammenfassung.

\lstinputlisting[language=python,linerange={1-3,73-93}, label={tests:lst:SimpleDocTestPrintFailure}]{chapters/basics/src/tests/simpleDocTest.py}




\subsection{unittest}
\label{tests:sec:UnitTest}
Zum Testen unter Python gibt es ein eigenes Modul.
Dieses wurde in Anlehnung an JUnit aus Java erstellt.
Ziel ist es, dem Programmierer zu erm�glichen kleine wiederholbare Tests zu schreiben.
Mit diesen Testf�llen l�sst sich Programmcode auf Integrations- und Operationsebene testen.


\subsubsection*{Beispieltest}
\label{tests:sec:BeispielTest}

Zum Einstieg in das Thema Unittest, zun�chst ein kleines Beispiel einer Testklasse.
\lstinputlisting[language=Python, label={tests:lst:BeispielTest}]{chapters/basics/src/tests/simpleTest.py}

Um einen Test zu erstellen, muss die Klasse, in der die Testf�lle definiert werden, von \lstinline$unittest.TestCase$ ableiten.
Hierf�r muss zuvor das Testframework \lstinline$unittest$ importiert werden.
Anschlie�end k�nnen die einzelnen Testmethoden einleitend mit der Bezeichnung \textit{test} definiert werden.
Die einzelnen Testf�lle werden parallel und ohne Reihenfolge abgearbeitet.
Zus�tzlich zu den einzelnen Testf�llen existieren noch zwei weitere Methoden zu Ablaufsteuerung.
Die Methode \lstinline$setUp()$ sowie die Methode \lstinline$tearDown()$.
Diese beide Methoden erlauben das Ausf�hren von Code vor und nach jeder einzelnen Testmethode.
Dies erm�glicht es bestimmte Testvoraussetzungen vor jedem Test schaffen.


\subsubsection*{Ausf�hren von Testzusammenstellungen}
\label{tests:sec:AusfuehrenVonTestzusammenstellungen}
Die erstellten Tests k�nnen nicht so ohne Weiteres auf der Konsole ausgef�hrt werden.
Um das Testen zu starten, ist es notwendig Python �ber den Parameter \lstinline$-m unittest$ aufzurufen.
Dies ist erforderlich um das Modul Unittest im Skriptmodus zu starten.
Im Anschluss folgt das zu testende Subjekt.
Dieses kann entweder mehrere Module, eine Klasse oder eine einzelne Methode sein.
Die Pythondatei selbst ist ebenfalls ein m�gliches Testsubjekt sein.
Hierbei werden nat�rlich nur die Testf�lle in der Datei ausgef�hrt.

\begin{lstlisting}[label=tests:lst:AusfuehrenVonTestzusammenstellungen,language=bash]
python -m unittest test_module1 test_module2
python -m unittest test_module.TestClass
python -m unittest test_module.TestClass.test_method
python -m unittest tests/test_something.py
\end{lstlisting}

Empfehlenswert ist des Weiteren noch der Parameter \lstinline$-v$ da dieser daf�r sorgt, dass die erf�llten Tests angezeigt werden.
Weitere Parameter hierzu k�nnen der Python Dokumentation \cite{pythondoku}.
Alternativ kann durch den Parameter \lstinline{-h} eine Liste der Parameter auf der Konsole ausgegeben werden.

\subsubsection*{Test Discovery}
Das Modul Unittest bietet seit Python 3.2 eine 

