% !TeX root = ../../pythonTutorial.tex
\section{Testing}
\label{testing:sec:Testing}
Zum Testen unter Python gibt es mehrere Module die das Testen unterst�tzen.
Zum einen das Modul \lstinline$doctest$, welches als interaktive Dokumentation Testmuster bereitstellt.
Zum anderen das Modul \lstinline$unittest$, welches den Unittests unter Java mit JUnit sehr �hnelt.
Beide Module erm�glichen Regressionstests.
\subsection*{doctest}
\label{testing:sec:DocTest}
Das Modul \lstinline$doctest$ erm�glicht es, Tests parallel mit dem Programmcode in die selbe Datei zu schreiben.
Dies ist sowohl innerhalb von Funktionen als auch au�erhalb m�glich.
Hierbei werden die Tests in Kommentarbl�cken (\lstinline$"""$) durch die Zeichenfolge \lstinline$>>>$ eingeleitet.
Danach wird die Funktion, die getestet werden soll, mit den gew�nschten Testparametern aufgerufen.
Diesem folgt dann entweder das Ergebnis oder eine Fehlerbehandlung im Falle einer zu pr�fenden Exception.

Das folgende Beispiel\randnotiz{Beispiel} zeigt, wie \lstinline$doctests$ in der Praxis Anwendung finden.
Die Tests werden, sobald der Code ausgef�hrt, wird ebenfalls ausgef�hrt. 
Dadurch das die Tests bei dem zu testenden Code stehen wird ein direkter Bezug zwischen beiden hergestellt.
Sollte beim Ausf�hren kein Test fehlschlagen, kommt es zu keiner Ausgabe durch den Interpreter.
Ist trotzdem eine Ausgabe erw�nscht, ist diese mit dem Parameter \lstinline$-v$ (verbose) aktivierbar.

\lstinputlisting[language=Python,lastline=37, label={tests:lst:SimpleDocTest}]{chapters/basics/src/tests/simpleDocTest.py}

Im nachfolgenden Beispiel ist die Ausgabe der Tests aus dem oben gezeigten Codebeispiel, zu sehen.
Der Ablauf, der einzelnen Test, ist jedes Mal gleich.
Zuerst wird der Test, mit den Versuchsparametern ausgegeben und im Anschluss der Erwartungswert gezeigt.
Sollte der R�ckgabewert der Funktion dem Erwartungswert entsprechen, wird der Test mit OK beendet.
Gleiches gilt beim Testen auf Exceptions.
Zum Schluss werden die Ergebnisse in einer Auflistung zusammengefasst und nach Zugeh�rigkeit gruppiert.
Hierbei handelt es sich um einen freien Test und drei innerhalb der Funktion.
Danach gibt es noch eine weitere Zusammenfassung, die die Tests nach Erfolg und Misserfolg gruppiert.
Dies soll dem Anwender erm�glichen, alle Tests mit einem Blick zu erfassen.

\lstinputlisting[language=python,linerange={1-3,40-69}, label={tests:lst:SimpleDocTestPrintSuccess}]{chapters/basics/src/tests/simpleDocTest.py}

Um den Fehlerfall zu betrachten, wurde die R�ckgabe der Funktion von \lstinline$x * y$ auf \lstinline$x + y$ ge�ndert.
Dies sorgte sofort f�r zwei Fehler beim Start des Programms.
Die einzelnen Fehler werden getrennt dargestellt.
Im Unterschied zum Erfolgsfall wird hier die Stelle angegeben, an der der Test steht.
Zus�tzlich dazu wird der zur�ckgegebene Wert angezeigt.
Am Ende erfolgt wieder eine Zusammenfassung.

\lstinputlisting[language=python,linerange={1-3,73-93}, label={tests:lst:SimpleDocTestPrintFailure}]{chapters/basics/src/tests/simpleDocTest.py}

�ber \randnotiz{Strukturierung} das Modul \lstinline$doctest$ ist es einerseits m�glich, Tests direkt an den Programmcode zu h�ngen und �ber den Docstring der Funktion Tests auszuf�hren.
Diese Tests werden am Ende eines Moduls mit \lstinline$doctest.testmod()$ aufgerufen.
Andererseits ist es M�glichkeit �ber \lstinline$doctest.testfile()$ separate Testdateien als Tests auszuf�hren.
Es ist also m�glich Tests zu einer Modul oder eine Klasse au�erhalb der Klasse zu definieren.
Damit der Python Interpreter wei� welches Modul hierbei getestet werden soll muss dies vorher �ber eine Importanweisung bekannt gemacht werden.
Da sich der Befehl in einem Docstring befindet, muss dieser wie die Funktionsaufrufen �ber \lstinline$>>>$ eingeleitet werden.
Bei dieser M�glichkeit der Strukturierung ist es dar�ber hinaus auch m�glich, die Tests unabh�ngig vom Programmcode ablaufen zu lassen.
Hierf�r muss wie im folgenden Listing zu sehen, das Modul \lstinline$doctest$ angegeben und die Testdatei aufgerufen werden.
Das Modul erkennt automatisch, dass es in diesem fall \lstinline$doctest.testfile()$ nutzen muss.
\begin{lstlisting}[label=testing:lst:AusfuehrenVonDoctests,language=bash]
    python -m doctest exampleTest.txt
\end{lstlisting}

Die Textdatei f�r ausgelagerte Testf�lle kann die wie folgt aussehen.

\lstinputlisting[label={tests:lst:docTestExtern}]{chapters/basics/src/tests/doctest.txt}

\tip{Grunds�tzlich ist die Menge des Texts, der zus�tzlich zum eigentlichen Test in einer ausgelagerten doctest-Datei steht, beliebig.
Zu beachten ist lediglich die Importanweisung sowie die Notation mit \lstinline$>>>$ zum Aufrufen der Funktion und deren Ausgabe im Anschluss.}