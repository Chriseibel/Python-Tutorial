% !TeX root = ../../pythonTutorial.tex
\section{Unit Test}
\label{unittest:sec:unittest}

Zum testen unter Python gibt es ein eigenen Testing Framework.
Dieses wurde in Anlehnung an JUnit aus Java erstellt.
Ziel ist es, dem Programmierer zu erm�glichen kleine wiederholbare Testf�lle zu Verf�gung zu stellen.
Mit diesen Testf�llen l�sst sich Programmcode auf Intergrations- und Operationsebene testen.

\subsection{Beispieltest}

Zum Einstieg in das Thema Testen, zun�chst ein kleines Beispiel einer Testklasse.
\lstinputlisting[language=Python, label={unittest:lst:beispieltest}]]{chapters/basics/src/unittest/simpleTest.py}

Um einen Test zu erstellen muss die Klasse in der die Testf�lle definiert werden sollen von \lstinline$unittest.TestCase$ ableiten.
Hierf�r muss zuvor das Testframework \lstinline$unittest$ importiert werden.
Anschlie�end k�nnen die einzelnen Testmethoden einleitend mit der Bezeichnung \textit{Test} definiert werden.
Die einzelnen Testf�lle werden paralell und ohne Reihenfolge abgearbeitet.
