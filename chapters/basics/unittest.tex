% !TeX root = ../../pythonTutorial.tex
\section{Unit Test}
\label{unittest:sec:UnitTest}
Zum Testen unter Python gibt es ein eigenes Modul.
Dieses wurde in Anlehnung an JUnit aus Java erstellt.
Ziel ist es, dem Programmierer zu erm�glichen kleine Wiederholbare Tests zu schreiben.
Mit diesen Testf�llen l�sst sich Programmcode auf Intergrations- und Operationsebene testen.


\subsection{Beispieltest}
\label{unittest:sec:BeispielTest}

Zum Einstieg in das Thema Testen, zun�chst ein kleines Beispiel einer Testklasse.
\lstinputlisting[language=Python, label={unittest:lst:BeispielTest}]{chapters/basics/src/unittest/simpleTest.py}

Um einen Test zu erstellen muss die Klasse in der die Testf�lle definiert werden sollen von \lstinline$unittest.TestCase$ ableiten.
Hierf�r muss zuvor das Testframework \lstinline$unittest$ importiert werden.
Anschlie�end k�nnen die einzelnen Testmethoden einleitend mit der Bezeichnung \textit{test} definiert werden.
Die einzelnen Testf�lle werden parallel und ohne Reihenfolge abgearbeitet.
Zus�tzlich zu den einzelnen Testf�llen existieren noch zwei weitere Methoden zu Ablaufsteuerung.
Die Methode \lstinline$setUp()$ sowie die Methode \lstinline$tearDown$.
Diese beide Methoden erlauben das ausf�hren von Code vor und nach jeder einzelnen Testmethode.
Dies erm�glicht bestimmte Variablen beziehungsweise Testzusammenstellungen vor jedem Test zu instantiieren.


\subsection{Ausf�hren von Testzusammenstellungen}
\label{unittest:sec:AusfuehrenVonTestzusammenstellungen}
Die erstellten Tests k�nnen nicht so ohne weiteres auf der Konsole ausgef�hrt werden.
Um das Testen zu starten ist es notwendig Python �ber den Parameter \lstinline$-m unittest$ aufzurufen.
Dies ist erforderlich um das Modul Unittest im Skriptmodus zu starten.
Im Anschluss folgt des zu testende Subjekt.
Dieses kann entweder mehrere Module, eine Klasse oder eine einzelne Methode sein.
Ebenfalls m�glich ist es den Pfad zu Pythondatei anzugeben.
Hierbei werden nat�rlich nur die Testf�lle in der Datei ausgef�hrt.

\begin{lstlisting}[label=unittest:lst:AusfuehrenVonTestzusammenstellungen,language=bash]
python -m unittest test_module1 test_module2
python -m unittest test_module.TestClass
python -m unittest test_module.TestClass.test_method
python -m unittest tests/test_something.py
\end{lstlisting}

Empfehlenswert ist des Weiteren noch der Parameter \lstinline$-v$ da dieser daf�r sorgt das die erf�llten Tests angezeigt werden.
Weitere Parameter hierzu k�nnen der Python Doku \cite{pythondoku}.
Alternativ kann durch den Parameter \lstinline{-h} eine Liste der Parametern auf der Konsole ausgegeben werden.




