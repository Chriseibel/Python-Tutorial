% !TeX root = ../../pythonTutorial.tex

\section{Grundlagen Zusammenfassung} 
\label{grundlagenzs:sec:Zusammenfassung}

In diesem Kapitel hat der Leser seine ersten Schritte mit der Sprache Python gemacht. 
Es wurde grundlegend vorgestellt, wobei es sich bei Python �berhaupt handelt und wie die Programmiersprache installiert wird. 
Es wurde der Python Interpreter vorgestellt, einer einfachen Konsolenanwendung zur Ausf�hrung von Python Code. 
Weiterhin wurden Kommentare und die Blockstruktur eingef�hrt. 
Um das in diesem Kapitel erlangte Wissen anzuwenden, wurde ein klassisches Hello-World! Programm geschrieben. 
Um einen guten Einstieg in die Programmierung zu bekommen, war ein erster �berblick �ber die verschiedenen IDE's n�tig. In dieser �bersicht sollte klar geworden sein, dass es eine Vielzahl von Entwicklungsumgebungen gibt, mit deren Hilfe man Python programmieren kann.
Anschlie�end wurden die unterschiedlichen einfachen Datentypen betrachtet. Dabei wurden neben den Zahlenwerten, Boolean- und Stringvariablen auch die Aufz�hlungen mit der Hilfe von ENUMs gezeigt. Dem Leser wurde ebenfalls gezeigt, dass es ein neutrales Element gibt, das an verschiedenen Stellen zum Einsatz gebracht werden kann. Im Zusammenhang mit einfachen Datentypen wurde auch eine Besonderheit von Python aufgezeigt, n�mlich dass eine Variable ein Objekt referenziert und damit keinem festen Typ zugeordnet wird.
Im Weiteren wurde der Umgang mit Abfragen gezeigt. Zum einen wurde dabei betrachtet wie man if-else Anweisungen einsetzt und wie alternativ dazu Conditional Expressions verwendet werden k�nnen. Zum anderen fiel der Blick auf die Benutzung von Schleifen in der Python-Programmierung. Zum Abschluss des Themas Kontrollstrukturen wurde noch kurz die allgemeine �bersicht zu den g�ngigen logischen Operatoren veranschaulicht.
%TODO ZSFASSUNG COLLECTIONS 


