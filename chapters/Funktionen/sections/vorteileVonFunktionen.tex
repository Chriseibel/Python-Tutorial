% !TeX root = ../../pythonTutorial.tex
\section{Vorteile von Funktionen}

Warum benutzt man eigentlich Funktionen? Diese bieten nun mal
eine Vielzahl an Vorteilen, wie z.B.

\begin{itemize}
	\item das Aufteilen von komplexen Aufgaben in mehrere simple.
	\item das verhindern von Code-Duplikationen.
	\item bessere Lesbarkeit/Erweiterbarkeit/Ver�nderbarkeit.
	\item vereinfachtes Debugging.
\end{itemize}

Im Folgenden werden die Punkte n�her betrachtet.

\subsection{Aufteilen von komplexen Aufgaben}

Bei komplexen Aufgaben kommt es schnell zu einer hohen Anzahl an
Code-Zeilen. Darunter leidet nicht nur der Programmierer, sondern auch Kollegen,
welche an dieser Aufgabe mitarbeiten, erweitern oder ver�ndern sollen.
Der Code ist schlecht zu lesen und die Fehlersuche die Nadel im Heuhaufen.
Schritte die immer wieder gebraucht werden, w�ren das Abfangen, Auslesen, Interpretieren
und Aufbereiten von Daten. Anf�ngern k�nnte es passieren, diese Aufgaben in einer
gro�en komplexen Funktion zu schreiben.

\begin{lstlisting}[caption=Schlechter Umgang mit komplexen Prozessen, label=badComplexFunction]
def processData(source):
  ...
  return finalData
\end{lstlisting}

Es ist mehr als m�hselig herauszufinden, was denn genau in dieser Funktion passiert.
Ein Beispiel zum Aufteilen des Code-Blocks, k�nnte da schon etwas mehr Licht ins Dunkel
bringen.

\begin{lstlisting}[caption=Komplexit�t aufgeteilt in mehrere Funktionen, label=goodComplexFunction]
def processData(source)
	rawData = readData(source)
	parsedData = parseData(rawData)
	editedData = editData(parsedData)
	finalData = sortData(editedData)
	return finalData
\end{lstlisting}

\subsection{Reduktion von Code-Duplikationen}

Gewisse Prozesse werden beim Programmieren immer wieder ben�tigt.
Beim Arbeiten mit Datens�tzen ist es �blich, diese nach gewissen Kriterien zu sortieren. 
In einer Datenbank ist die Sortierung nach der Identifikationsnummer von gro�em Vorteil.
Nun m�chte man nicht f�r jeden Datensatzaufruf, diese \textbf{Funktion} duplizieren.
Daher speichert man diesen Prozess in einer Funktion, auf welche man von verschiedenen
Positionen im Programm zugreifen kann.

\subsection{Bessere Lesbarkeit, Erweiterbarkeit, Ver�nderbarkeit}

Wie im Unterabschnitt \textit{Aufteilen von komplexen Aufgaben} zu sehen ist,
bringt das Aufteilen des Codes in spezifische Funktionen eine bessere Lesbarkeit mit sich.
Der Nutzer muss den Code nicht erst interpretieren, bei intelligent gew�hlten 
Funktionsnamen, versteht er, was in der Funktion passiert. Besonders beim Debuggen 
kann das gro�e Vorteile mit sich bringen, da der Programmierer nicht an den falschen 
Stellen zu suchen braucht.

array = calculateArray()

sortedArray = quickSort(array)

In diesem Beispiel wei� der Nutzer, dass das Array durch einen QuickSort-Algorithmus sortiert wird.
Sollte nun auffallen, dass es sich um falsche Werte handelt, dann muss der Programmierer
nur in die calculateArray-Funktion sehen, sind die Werte falsch sortiert, so wird die quickSort-Funktion
n�her betrachtet.

Durch das Kapseln von Prozessen in einzelnen Funktionen, sind diese auch einfach erweiterbar und ver�nderbar.
W�re der Code nur dupliziert worden, m�sste man diesen an allen Stellen �ndern.
Da das Programm aber an diesen Stellen nur die Funktion aufruft, muss nur
diese Funktioniert erweitert oder ver�ndert werden.
