In diesem Kapitel gehen wir auf das Erstellen von grafischen Benutzeroberfl�chen mithilfe von
Tkinter in Python ein, welches unter macOS und Windows zum Lieferumfang von Python geh�rt.
Neben dem hier behandelten GUI-Toolkit existieren dennoch weitere,
unter anderem WxPython, PyQt und PySide, PyGTK, Kivy sowie PyFLTK.

\section{Tkinter}
\label{ui:einleitungTkinter:section:tkinter}

Tk ist ein freies, plattform�bergreifendes GUI-Toolkit von Scriptics
(fr�her von Sun Labs entwickelt) zur Programmierung von grafischen
Benutzeroberfl�chen (GUIs). Urspr�nglich f�r die Programmiersprache Tcl
(Tool command language) entwickelt, existieren heute Anbindungen an diverse
Programmiersprachen.
Unter vielen dynamischen Sprachen ist Tk das Standard-GUI und kann umfangreiche,
ab dem Release 8.0 mit nativem Look-and-Feel versehene Anwendungen erstellen,
die unver�ndert unter Windows, macOS und Linux laufen.

Tkinter ist eine Sprachanbindung f�r das am h�ufigsten verwendete GUI-Toolkit
Tk f�r die Programmiersprache Python.
Der Name steht als Abk�rzung f�r Tk interface. Tkinter war das erste GUI-Toolkit
f�r Python, weshalb es inzwischen auf macOS und Windows zum Lieferumfang
von Python geh�rt.

Tkinter besteht aus einer Reihe von Modulen. Das Tk interface wird von einem
bin�ren Erweiterungsmodul namens \lstinline$_tkinter$ bereitgestellt. Dieses Modul enth�lt
die Low-Level-Schnittstelle zu Tk und sollte niemals direkt von
Anwendungsprogrammierern verwendet werden. Es handelt sich in der Regel um eine
Shared Library (oder DLL), kann aber in einigen F�llen statisch mit dem
Python-Interpreter verkn�pft sein \cite{tkinter}.
